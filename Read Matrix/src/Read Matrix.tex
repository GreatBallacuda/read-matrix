\documentclass{ctexart}
\usepackage{color}
\usepackage{fancyhdr}
\usepackage[a4paper, left=1in, right=1in, top=1in, bottom=1in]{geometry}
\usepackage{graphicx}

\pagestyle{fancy}
\fancyhf{}
\renewcommand{\headrulewidth}{0pt}
\renewcommand{\footrulewidth}{0pt}
\lhead{百分之三的Matrix}
\rhead{\thepage}

\definecolor{green}{rgb}{0.0, 0.25, 0.0}
\newcommand{\myparsep}{\noindent \rule[0.5ex]{\linewidth}{1pt}}
\newenvironment{myquote}{\color{green} \setlength{\leftskip}{6em} \setlength{\rightskip}{4em} \setlength{\parindent}{-2em}}{\par}

\hyphenation{ani-ma-trix}

\begin{document}
\centerline{\bf \fontsize{15.75pt} \baselineskip \selectfont The Matrix(第一部)非官方解释}
\vspace{12pt}
\centerline{作者:neverwin}
\centerline{Version 0.91 2006-08-03}
\centerline{重制:woctordho}
\centerline{2017-01-26}
\vspace{12pt}

时间是个有趣的东西,它改变了一些东西,却始终改变不了另外一些东西。

时间改变了我对Matrix的理解,却始终改变不了我对它的热爱。

我决定开始做一个很麻烦的事情:按照电影镜头的顺序,从头开始解释整个系列三部片子(以及动画版)。

每部电影都有“镜头语言”。大话西游里,至尊宝和紫霞在市集上,紫霞向他表白,他很不情愿地说了yes,我们看到他们背后的墙上是大量的京剧脸谱,这也是“电影语言”的一种。

我们对Matrix“吹毛求疵”的意义在于,Matrix的制作者非常聪明,他们讲故事只讲一遍。Neo为什么可以干掉章鱼?只有Oracle很简单的一句:“the One把他的能力从Matrix扩展到了这个世界。”这与其说是原因,不如说是个结论。探寻why,“追求答案”,这就是我们在这里的原因。

第一部Matrix是传奇,第一次……但是它的故事为了迎合观众的口味,还是不能100\%脱俗。

四年后的第二部彻底改变了一切。第二部Matrix是真正的Matrix,导演做了他们真正想做的事。看罢第二部,幡然醒悟,自己原来完全不懂第一部。

第三部Matrix在场面上只能打70分,但是在故事的完整性和深度上,并没有给整个系列减分。

只喜欢第一部的朋友,喜欢的只是电影本身,而这里讨论的是Matrix。Matrix的意义在于,抛开电影,你想到了什么。

我们永远都只是在路上,执着是唯一的方向……

\myparsep

\begin{myquote}
(手机响)

Call trans opt: Received. 2-19-98 13:24:18 REC:Log$>$
\end{myquote}

这里出现的是Matrix的管理界面,机器在窃听Trinity和Cypher的对话,寻找Trinity的位置。同样的界面在Animatrix的“鬼屋”那集也出现了,那里机器发现Matrix里的一个地方出错了。

这个界面在Matrix第一部的结尾也出现了,只是在最后“死机”了。这是在告诉我们,Neo的出现,给Matrix带来了致命的错误。

\begin{myquote}
Cypher: Yeah.

Trinity: Is everything in place?

Cypher: You weren't supposed to relieve me.

Trinity: I know, but I want to take your shift.

Cypher: You like watching him, don't you? You like watching him.

Trinity: Don't be ridiculous.

Cypher: We're going to kill him. Do you understand that?

Trinity: Morpheus believes he is the One.

Cypher: Do you?

Trinity: It doesn't matter what I believe.

Cypher: You don't, do you?

Trinity: Did you hear that?

Cypher: Hear what?

Trinity: Are you sure this line is clean?

Cypher: Yeah, of course I'm sure.

Trinity: I'd better go.
\end{myquote}

这里,我们明白了几件事:

1、Trinity在这里己经对Neo有好感了,就算不是“爱”,也算小“暗恋”了。

2、Cypher在这里己经开始出卖他们,Smith们后来也说“线人消息没错”,就是在说这里。

3、他们在寻找the One,可是如果找到的不是the One,机器就会把那个“人”杀了。

4、Morpheus“相信”Neo就是the One。为什么?这里没有说。

最后机器追踪出来的数字是5550690,这是什么意思呢?呵,555是代表魔鬼的意思,69在图形上是个八卦,代表的就是平衡。

\myparsep

\begin{myquote}
(Hotel room)
\end{myquote}

\begin{figure}[htb]
\centering
\includegraphics[width=0.5\linewidth]{fig/read_Matrix-1}
\end{figure}

这是第一个“Matrix里”的镜头,我们先看到手电筒的光,然后看到了警察。Matrix里的“光”代表能量。警察控制了光,代表机器控制了Matrix。

上面的镜头是Trinity的303房间,303和Neo的房间101,还有Animatrix里那个“失败的the One”——侦探,他的房间是201,这些都是有含义的。

\begin{myquote}
Cop: Freeze, police! Hands on your head! Do it! Do it now!

(Street)

Agent Smith: Lieutenant...

Lieutenant: Oh, shit.

Agent Smith: Lieutenant, you were given specific orders.

Lieutenant: Hey, I'm just doing my job. You give me juris --- my diction crap, you can cram it up your ass.

Agent Smith: Your orders were for your protection.

Lieutenant: I think we can handle one little girl ... I sent two units. They're bringing her down now.

Agent Smith: No, Lieutenant, your men are already dead.
\end{myquote}

Smith出场了,看起来很cool,呵。

他手下也是两个人,三人组合。Neo、Trinity和Morpheus也是三人组合,后来第三部Neo被关在车站,Oracle就给了他们自己的“保镖”Seraph,又是一个三人组合。第二部炸电厂,也是三组,每组三个人。

电影里的3代表的是“能完成任务的”、“稳定的”结构。3在几何学里代表三角形,我们都知道,三角形是最稳定的结构。

\begin{myquote}
(Hotel room)
\end{myquote}

这里是Trinity干掉警察的场景,除了感慨她当年还不显“老”外,还要说的是请记住她的三个标志动作:

1、Trinity之飞踢,Shrek里面都抄这个动作。她把那个警察踢到了墙里,代表T。

2、她上墙的动作,在墙上跑了半圈,躲子弹,代表R。

3、她高踢腿,踢背后的敌人的头,代表I。

到了第三部的Hell Club里gun check那场戏,这三个动作一样都出现了。

\begin{myquote}
Trinity: Morpheus, the line was traced. I don't know how.

Morpheus: I know. They cut the hard line. There's no time. You're going to have to get to another exit.

Trinity: Are there any agents?

Morpheus: Yes.

Trinity: God damn it.
\end{myquote}

God damn it,中文当然直接翻译成见鬼、该死。但是导演在这个电影里大量地使用了“一语双关”,来表达他们的思想。

对于这些“人”来说,不管他们是否被拔了插头,他们都不过是机器控制下的“棋子”,所以机器的老大就是他们的上帝,他们的God有“诅咒”的意思。这里的God damn it,可以理解成“机器故意(编程)安排的”。

电影里大量地使用了God damn it,每个地方你都能用这个意思套上理解。有时用的是your own damn mind,这里就说的是“你自己拿主意的事”了。

\begin{myquote}
Morpheus: You have to focus, Trinity. There's a phone at Wells and Lake. You can make it.
\end{myquote}

Wells and Lake,井和湖,虽然它们也是芝加哥的路名(导演说他们命名道路都用他们老家的地名),但是这两个名字都是指“有水的地方”。在圣经和很多其它宗教里,“水”指的都是“救命的东西”。

\begin{myquote}
Trinity: All right.

Morpheus: Go.

(Rooftop)

Cop: That's impossible.

(Building)
\end{myquote}

\begin{figure}[htb]
\centering
\includegraphics[width=0.5\linewidth]{fig/read_Matrix-2}
\end{figure}

Trinity飞过那个窗户。看上面的图,我们看到破烂的窗户,居然刚好是个“3点钟”。旁边是盏很亮的灯,联系我前面说的Matrix里光的含义,很容易理解。

\begin{myquote}
Trinity: Get up, Trinity. Just get up. Get up.
\end{myquote}

Trinity在这里说了三遍get up,一语双关。Get up可以解释成起来,也可以是“起床”的意思,或者说是“醒来”。

后来Morpheus被营救时,Trinity也喊了三遍get up。

她对着躺在飞船上接入Matrix的Neo也喊了三遍get up。

这都不是偶然的。

\begin{myquote}
(Street)

Agent Brown: She got out.

Agent Smith: It doesn't matter.

Agent Jones: The informant is real.

Agent Smith: Yes.

Agent Jones: We have the name of their next target.

Agent Brown: The name is Neo.

Agent Smith: We'll need a search running.

Agent Jones: It has already begun.
\end{myquote}

\myparsep

\begin{myquote}
(Neo's apartment)
\end{myquote}

\begin{figure}[htb]
\centering
\includegraphics[width=0.5\linewidth]{fig/read_Matrix-3}
\end{figure}

这里是Neo和“电脑”聊天的镜头。

除了叫他“wake up”之外,Trinity(或者是Morpheus)说:“The Matrix has you ...”Neo按的是Ctrl+X,在Unix里,这是退出指令。可是这又是一个一语双关的例子,读起来就是control ex(it)。

然后对方说:“Follow the white rabbit ...”Neo按了两次Esc,“两次Esc”=“two escape”=“to escape”,所以连起来就是“follow the white rabbit to escape”。电影强就强在这里,你可以不知道这些,但是如果你知道了,你就和我一样“明白”了。

\begin{myquote}
Neo: What? What the hell? ... follow the white rabbit? ... who is it?

Choi: It's Choi.

Neo: Yeah, yeah. You're two hours late.

Choi: I know. It's her fault.

Neo: Got the money?

Choi: Two grand.

Neo: Hold on.

Choi: Hallelujah. You're my savior, man. My own personal Jesus Christ.

Neo: You get caught using that ...

Choi: Yeah, I know. This never happened. You don't exist.

Neo: Right.

Choi: Something wrong, man? You look a little whiter than usual.

Neo: My computer, it ... you ever have that feeling where you're not sure if you're awake or still dreaming?

Choi: Mm, all the time. It's called mescaline. It's the only way to fly. Hey, it just sounds to me like you need to unplug, man. You know, get some R and R. What do you think, Dujour? Shall we take him with us?

Dujour: Definitely.

Neo: I can't. I have work tomorrow.

Dujour: Come on, it'll be fun. I promise.

Neo: Yeah, sure. I'll go.
\end{myquote}

\begin{figure}[htb]
\centering
\includegraphics[width=0.5\linewidth]{fig/read_Matrix-4}
\includegraphics[width=0.5\linewidth]{fig/read_Matrix-4-1}
\includegraphics[width=0.5\linewidth]{fig/read_Matrix-4-2}
\end{figure}

这里可以讨论的东西不少。

1、来找Neo的人,男的叫Choi(choice,选择),女的叫Dujour(法语,一天之中的),他们合起来就是choice of the day。

这个人要的到底是什么?在原始剧本里,说他的车被警察锁在路边,Neo黑进警察局的数据库,结果很快一个警察就来给那个人的车开锁。在电影里,没有说到底因为什么,只是说Neo给了他一个光盘,也许是Windows 98的完美破解程序。

2、Neo拿来放光盘的书叫Simulacra and Simulations,作者叫Jean Baudrillard,放光盘的那章叫On Nihilism(论虚无主义)。Amazon上这本书卖疯了,都是被Matrix的爱好者抢光的。

3、从“白兔”从电脑屏幕出现,我们开始看到大量的“爱丽丝漫游奇境”在电影里的投影。这本书我小时候看过,觉得只是不错的童话,看了Matrix以后重新看,发现它里面包含了大量只有成人才能理解的哲学道理。

Neo就和Alice—样,跟着“白兔”开始钻“兔子洞”了。

这个场景我想多解释两句。大家想过吗,为什么Morpheus他们知道Choi和Dujour要来找Neo,而且知道他们要去那个俱乐部?

应该这样想,Trinity他们一直在“监视”Neo,他们看到那堆人来找Neo,所以Trinity就说,让Neo跟着他们走。如果他们要去看电影,Trinity就会在电影院出现了。

4、这里的对话里也有很多“一语双关”的东西:“you don't exist”“unplug”“savior”……

\myparsep

\begin{myquote}
(Club)

Trinity: Hello, Neo.

Neo: How do you know that name?

Trinity: I know a lot about you.

Neo: Who are you?

Trinity: My name is Trinity.

Neo: Trinity. The Trinity? That cracked the IRS D-base?

Trinity: That was a long time ago.

Neo: Jesus.

Trinity: What?

Neo: I just thought, um ... you were a guy.

Trinity: Most guys do.

Neo: It's you on my computer. How did you do that?

Trinity: Right now all I can tell you is that you're in danger. I brought you here to warn you.

Neo: Of what?

Trinity: They're watching you, Neo.

Neo: Who is?

Trinity: Please just listen. I know why you're here, Neo. I know what you've been doing. I know why you hardly sleep, why you live alone, and why night after night you sit at your computer. You're looking for him. I know, because I was once looking for the same thing. And when he found me, he told me I wasn't really looking for him. I was looking for an answer. It's the question that drives us mad. It's the question that brought you here. You know the question just as I did.

Neo: What is the Matrix?

Trinity: The answer is out there, Neo. It's looking for you. And it will find you, if you want it to.
\end{myquote}

\begin{figure}[htb]
\centering
\includegraphics[width=0.5\linewidth]{fig/read_Matrix-5}
\end{figure}

首先出现的是个跳舞的场景。三部Matrix都有劲歌狂舞的场面,后面我比较地来解释。这里先说明,这里是“Matrix里的人在跳舞”。你看Choi和Dujour坐在那里,和第三部的法国人和他老婆的场景像吗?

Trinity出现了,她说的“找来找去”那段话很有意思。她说是“问题在驱使着我们去寻找着答案”。在第二部,Smith说“目的在驱使我们”。这也是两部电影不同的地方,第一部是在“问问题”,第二部是在“揭示目的”。

\myparsep

\begin{myquote}
(Neo's apartment)

Neo: Oh, shit. Oh, shit, shit.
\end{myquote}

Neo在9:18分醒来。有网友在网上聊天里问导演兄弟,9:18代表的意思。他们回答是,这是他(不知道是Andy还是Larry)老婆的生日。

\begin{myquote}
(Office)

Mr.~Rhineheart: You have a problem with authority, Mr.~Anderson. You believe that you are special, that somehow the rules do not apply to you. Obviously, you are mistaken. This company is one of the top software companies in the world, because every single employee understands that they are part of a whole. Thus if an employee has a problem, the company has a problem. The time has come to make a choice, Mr.~Anderson. Either you choose to be at your deskon time from this day forward, or you choose to find yourself another job. Do I make myself clear?

Neo: Yes, Mr.~Rhineheart. Perfectly clear.
\end{myquote}

\begin{figure}[htb]
\centering
\includegraphics[width=0.5\linewidth]{fig/read_Matrix-6}
\end{figure}

这里Neo在被他的老板训话。他老板叫Rhineheart,Rhine是莱茵河,可以指欧洲,也可以指德国。电影里一直出现的那个旅馆叫Heart o' the City(这里没有用of,用的是o',很奇怪),heart在电影里似乎有特别的意思,还没想明白。

这段就是电影里的重头戏,很多人不喜欢电影里的冗长的对话,但是这些地方才给观众解释了Matrix.

\begin{figure}[htb]
\centering
\includegraphics[width=0.5\linewidth]{fig/read_Matrix-7}
\end{figure}

首先是窗外的工人在洗窗户,这有两个意思:一是电影里的“雨”在这里第一次出现了,它代表的是the One道路里重要的部分;二是擦窗户的声音,吱吱呀呀的,在电影里,这种声音,还有火车刹车的声音,都代表着“必然性”、“不可避免的”。

Neo老板的话,全部都是“一语双关”,表面那层意思我们都知道了,讲背后的意思:

You have a problem with authority. 你不服管,你想摆脱控制。

You believe that you are special. The One还不特别吗?

... that somehow the rules do not apply to you. 因为Neo会打破“rule”(规则)。

This company is one of the top software companies in the world, because every single employee understands that they are part of a whole. Thus if an employee has a problem, the company has a problem. 螺丝钉理论,意思就是说就算Matrix是个很大很复杂的系统,但是错误却总是来自其中任何一个微小的部分,小问题带来大灾难。

The time has come to make a choice, Mr.~Anderson. 哈,叫Neo做选择啦。

Either you choose to be at your desk on time from this day forward, or you choose to find yourself another job. 让他当the One去了。

\myparsep

\begin{myquote}
Fedex man: Thomas anderson?

Neo: Yeah, that's me.

Fedex man: Ok, great. Have a nice day.

Neo: Hello.

Morpheus: Hello Neo. Do you know who this is?

Neo: Morpheus.

Morpheus: Yes. I've been looking for you, Neo. I don't know if you're ready to see what I want to show you, but unfortunately you and I have run out of time. They're coming for you, Neo, and I don't know what they're going to do.

Neo: Who's coming for me?

Morpheus: Stand up and see for yourself.

Neo: What, right now?

Morpheus: Yes, now. Do it slowly. The elevator.
\end{myquote}

Neo终于听到Morpheus的声音了,可是却是坏消息,Morpheus让他起来看电梯那边的agent。这里我讲解一下电影里“电梯”的含义。

电梯除了叫elevator,也叫lift(抬起,升起的意思)。电影里的“电梯”通通都是只能上,不能下的。

这里,Neo没法用电梯“下去”逃跑。

后来Neo和Morpheus用电梯“上去”见Oracle. Neo和Trinity用电梯“上去”救Morpheus。第二部里,Neo他们用电梯“上去”法国人的餐馆。

破坏完电路系统,Trinity想用电梯“下去”离开,被干探杀掉。

第三部里,Seraph他们用电梯“下去”法国人的俱乐部,这里不是“下去”?别忘了,法国人的那个地方规则是颠倒的。

后来Seraph和小女孩Sati想用电梯“下去”,电梯不灵了。

\begin{figure}[htb]
\centering
\includegraphics[width=0.5\linewidth]{fig/read_Matrix-8}
\end{figure}

而且这里电梯那里的那个exit标志是绿色的,绿色代表机器的控制。“机器的出口”,Neo当然不能选。

\begin{myquote}
Neo: Oh, shit.

Morpheus: Yes.

Neo: What the hell do they want from me?

Morpheus: I don't know, but if you don't want to find out, I suggest you get out of there.

Neo: How?

Morpheus: I can guide you, but you must do exactly as I say.

Neo: Ok.

Morpheus: The cubicle across from you is empty.

Neo: What if they ...

Morpheus: Go, now ... stay here for just a moment. When I tell you, go to the end of the row, to the office at the end of the hall. Stay as low as you can. ... go, now. ... good. Now, outside there is a scaffold.
\end{myquote}

这里有个有趣的地方,Morpheus在教Neo跑路时,我们看到了公司里的一个标志。(我的图不是很清楚,在后面的墙上,大家自己看片)

\begin{figure}[htb]
\centering
\includegraphics[width=0.5\linewidth]{fig/read_Matrix-9}
\end{figure}

这里的标志是meta cortechs,对照前面的大楼标志metacortex。有人说这是电影的错误, 当然不是。少打一个s可以说是错误,故意在meta后面加空格,就不会是错误。

这样来解释,把词拆开看,meta cor-techs。

techs指技术人员(technicians),前缀cor是指“一窝”,它们连起来就是一窝技术员,现实的例子就是mop里苦命的程序员联盟。

meta这个词在哲学里的意思很有意思,只可意会,不可言传。举例来说,meta-language,就是指“用来解释语言的语言”,所以meta就相当于嵌套的意思。

全部连起来解释就是,这个metacortex公司是“用来产生程序的程序”。

天啊,这就是我说的这个帖子写不完的原因,讲了半天,我才写到电影的15分钟……

顺便再提一下数字“3”的应用。Neo的办公桌上,三个文件夹子,他的文件柜也是三层的。

\begin{figure}[htb]
\centering
\includegraphics[width=0.5\linewidth]{fig/read_Matrix-10}
\end{figure}

Neo在办公室里逃窜时还有一个有趣的细节:看下面的图,他跑过一个同事时,那哥们在复印东西,这里再次提到Matrix里“光”的意义。

\begin{figure}[htb]
\centering
\includegraphics[width=0.5\linewidth]{fig/read_Matrix-11}
\end{figure}

\begin{myquote}
Neo: How do you know all this?

Morpheus: We don't have time, Neo. To your left there's a window. Go to it ... open it. You can use the scaffold to get to the roof.

Neo: No way, no way. This is crazy.

Morpheus: There are two ways out of this building. One is that scaffold, the other is in their custody. You take a chance either way. I leave it to you.

Neo: This is insane. Why is this happening to me? What did I do? I'm nobody. ... shit. ... I can't do this.
\end{myquote}

Neo跑到这个房间以后,Morpheus让他爬出去,上脚手架,然后上屋顶。这个过程和Animatrix 里Neo让Kid做的几乎一样,可是人家Kid就办到了。

Neo选择了left(左边)的窗户。英文里的right(右边)也有“正确”的意思,所以电影里,只有选择右边的门才是“对”的,是符合机器安排的。

这里我也想讲一下Matrix里“选择”的规则。电影里的“选择”不像我们生活中的选择,比如有5个MM让你选择。电影里都是2项选择,就是“01”选择,“one way or the other”、“either way”。这也符合计算机的原理,因为就算是5个MM的选择,我们也可以用“选择闺女A,或者闺女BCDE之一”“选择闺女B,或者闺女CDE之一”这样的多个“01”选择来替代。这也是哲学原理中简化复杂现象的一个很基本的方法。

\begin{myquote}
(Street)

Trinity: Shit.
\end{myquote}

Neo被逮了,这里我结合几个截图,讲一下Matrix中的一个颜色应用规则。

\begin{figure}[htb]
\centering
\includegraphics[width=0.5\linewidth]{fig/read_Matrix-12}
\includegraphics[width=0.5\linewidth]{fig/read_Matrix-13}
\includegraphics[width=0.5\linewidth]{fig/read_Matrix-14}
\end{figure}

这三个镜头是Neo被干探们带走时,电影出现的关于Matrix的“长镜头”。你在背景里看到了什么吗?

没错,{\color{red} \bf 红色~}

不是很多,但是星星点点,我们总能看到。在几乎所有的Matrix场景里,包括三部电影,总有红色的“点”。我之前的文章里专门讨论过Matrix的颜色,这里简单点说,红色代表真理、真像。Matrix的整个色调是绿色的,代码世界,但是这不代表Matrix就是“完全虚假的”、“骗人的”,它依然有很多“道理”需要我们去探寻,去理解。电影里到处都是这样的例子,大家可以自己注意一下,我后面就不再重复了。

\myparsep

\begin{myquote}
(Interrogation)

\begin{figure}[htb]
\centering
\includegraphics[width=0.5\linewidth]{fig/read_Matrix-15}
\end{figure}

Agent Smith: As you can see, we've had our eye on you for some time now, Mr.~Anderson. It seems that you've been living two lives. In one life, you're Thomas A. Anderson, program writer for a respectable software company. You have a social security number, you pay your taxes, and you help your landlady carry out her garbage. The other life is lived in computers, where you go by the hacker alias Neo and are guilty of virtually every computer crime we have a law for. One of these lives has a future, and one of them does not. I'm going to be as forthcoming as I can be, Mr.~Anderson. You're here because we need your help. We know that you've been contacted by a certain individual, a man who calls himself Morpheus. Now whatever you think you know about this man is irrelevant. He is considered by many authorities to be the most dangerous man alive. My colleagues believe that I am wasting my time with you, but I believe that you wish to do the right thing. We're willing to wipe the slate clean, give you a fresh start, and all that we're asking in return is your cooperation in bringing a known terrorist to justice.

Neo: Yeah. Wow, that sound like a really good deal. But I think I got a better one. How about I give you the finger ... and you give me my phone call.

Agent Smith: Um, Mr.~Anderson. You disappoint me.

Neo: You can't scare me with this gestapo crap. I know my rights. I want my phone call.

Agent Smith: Tell me, Mr.~Anderson, what good is a phone call if you're unable to speak ... you're going to help us, Mr.~Anderson, whether you want to or not.
\end{myquote}

Neo和Smith第一次见面了。镜头开始是“电视墙”,很容易和第二部Architect房间里的那个“电视墙”联系起来。所以我们可以推理说,这个镜头代表Matrix的控制AI在“监视”他们的行动。就是说Architect他们一开始就把Neo当the One,而Smith没有,他只是在做“他应该做的”(遵循一个程序的规则)。

Smith和Neo说话前,他把墨镜拿掉了。电影里,每个人(程序)要敞开心扉说话(或者kiss)时,他们都摘墨镜。

Smith说的话也很有意思,我解释一些字面意思之外的意思:

It seems that you've been living two lives. 他说这个的时候,从镜头里的Smith的眼镜里,我们看到了两个Neo,这是电影里经常采用的“镜子”的摄影手法。

\begin{figure}[htb]
\centering
\includegraphics[width=0.5\linewidth]{fig/read_Matrix-16}
\end{figure}

One of these lives has a future, and one of them does not. 普通电影里,我们看到这句,理解肯定是,Neo当黑客没前途;可是在这里,电影的意思是,当程序员没前途,当the One那份很有前途的黑客职业吧。

I'm going to be as forthcoming as I can be. Smith自己也准备好干点“他应该干的”事情了。(我说了,我这里说的是字面外的意思,一语双关)

We know that you've been contacted by a certain individual, a man who calls himself Morpheus. Individual这个词,当然有“个人”的意思,但也有“单独的”、“个别的”的意思。用在这里,表示Morpheus这样拔了插头的家伙实在是异类。

Tell me, Mr.~Anderson, what good is a phone call if you're unable to speak? 这里出现了电影的一个主题思想,一个重要规则:“你没有嘴,你要电话干嘛?”就像后面的“你没有选择,你选择什么”一样,是在告诉我们,我们脑子的一些错误思想,基础就是不对的,现在到了该颠覆它们的时候了。这里,Neo要打电话(给律师),就像我们知道的一样,很多好莱坞电影也有这样的镜头,被警察关,打电话给律师。可是你突然没嘴了,你怎么打电话,怎么找律师?这里表达的思想很深刻,理解它,才能理解整个电影。

\myparsep

\begin{myquote}
(Neo's apartment)

Morpheus: This line is tapped, so I must be brief. They got to you first, but they've underestimated how important you are. If they knew what I know, you'd probably be dead.

Neo: What are you talking about? What ... what is happening to me?

Morpheus: You are the One, Neo. You see, you may have spent the last few years looking for me, but I've spent my entire life looking for you. Now do you still want to meet?

Neo: Yes.

Morpheus: Then go to the Adam Street Bridge.
\end{myquote}

Morpheus说,如果干探他们知道Neo是the One,他可能早死了。这里,我想大家应该明白一件事,机器方,或者说干探他们,根本不是要干掉the One,他们是在干掉那些“不称职”的the One。他们也和Morpheus他们一样,在“寻找”the One。Smith在片子最后杀死Neo,不过是Neo变成the One的一个步骤,是程序安排好的。

比如Cypher开始出卖他们,Smith压根就没问他the One的事情,他只关心Zion主机的密码。

所以,我想Morpheus说的话没错,但不是他的意思。应该说,如果Neo不是the One,他就死定了,就像Animatrix里的那个侦探。

Morpheus让Neo去Adam Street Bridge,Adam就是亚当,大家知道吧,他是世界的第一个人,代表开始。Bridge就是Morpheus代表的作用,他代表的数字是2,代表一种连接的关系,他引导着Neo进了the One的门。

\begin{myquote}
(Car)

Trinity: Get in.

Neo: What the hell is this?

Trinity: It's necessary, Neo. For our protection.

Neo: From what?

Trinity: From you. Take off your shirt.

Neo: What?

Switch: Stop the car. Listen to me, copper-top. We don't have time for twenty questions. Right now there's only one rule, our way or the highway.
\end{myquote}

Neo等他们的地方的墙上,有很多“涂鸦”,很抱歉,我对这些东西很不敏感,基本看不大明白。比如这里,Neo背后有句“bank is wrong”,银行错了?因为没有给导演他们贷款拍电影?大家可以补充。

Switch说的话很强。她称呼Neo是“copper-top”,铜盖子,那不就是电池吗?

Switch的话也说了另外一个Matrix里的规则:high(高)是机器走的,low(低)是这些“黑客”走的。

前面Morpheus教Neo逃命时,他说“stay as low as you can”,然后Neo也是弯着腰爬出去的。

第二部里,Morpheus说highway很危险。

第三部里,Seraph、Morpheus和Trinity经过一个铁路桥,上面的标牌是“lower clearance”。

大家记住low和high的区别。

Switch还说“我们没有时间(让你问)20个问题。”电影演到这里,Neo不多不少,正好问了20个问题!

\begin{myquote}
Neo: Fine.

Trinity: Please, Neo. You have to trust me.

Neo: Why?

Trinity: Because you have been down there, Neo. You know that road. You know exactly where it ends. And I know that's not where you want to be ... Apoc, lights. Lie back. Lift up your shirt.

Neo: What is that thing?

Trinity: We think you're bugged ... try and relax ... come on. Come on.

Switch: It's on the move.

Trinity: Shit.

Switch: You're going to lose it.

Trinity: No, I'm not. Clear.

Neo: Jesus Christ, that thing's real?
\end{myquote}

\begin{figure}[htb]
\centering
\includegraphics[width=0.5\linewidth]{fig/read_Matrix-17}
\end{figure}

这里我们第一次看到Trinity帮助Neo走the One道路。

Neo要下车了,Trinity拉住他:“你来过这里,你认识路,你知道路将如何结束,你不想再走老路。”她指的是Matrix,说Neo你不想再回Matrix吧。Neo被她劝住,回头了。

Trinity说“We think you're bugged.”这个bug既是放虫子,也是被监听的意思,所以是个“一语双关”的例子。

Jesus Christ, that thing's real? 我们可以把逗号拿开,翻译起来就是“真有耶稣啊?”。电影里耶稣指的是“救世主”(the One)。

\myparsep

\begin{myquote}
(Lafayette hotel)

Trinity: This is it. Let me give you one piece of advice. Be honest. He knows more than you can imagine.

Morpheus: At last. Welcome, Neo. As you no doubt have guessed, I am Morpheus.

Neo: It's an honor to meet you.

Morpheus: No,the honor is mine. Please, come. Sit down. I imagine that right now you're feeling a bit like Alice, tumbling down the rabbit hole? Hm?
\end{myquote}

伟大的Morpheus和神奇的Neo会面啦~记住这里他们的对话吧,第三部最后他们分开时,重复了这里的对话。

这里导演干脆明说了,电影里用Alice漫游仙境的故事,把Neo寻找Matrix比喻成打兔子洞。

\begin{myquote}
Neo: You could say that.

Morpheus: I can see it in your eyes. You have the look of a man who accepts what he sees because he is expecting to wake up. Ironically, this is not far from the truth. Do you believe in fate, Neo?

Neo: No.

Morpheus: Why not?

Neo: Because I don't like the idea that I'm not in control of my life.

Morpheus: I know exactly what you mean. Let me tell you why you're here. You're here because you know something, what you know you can't explain. But you feel it. You've felt it your entire life, that there's something wrong with the world. You don't know what it is, but it's there, like a splinter in your mind driving you mad. It is this feeling that has brought you to me. Do you know what I'm talking about?

Neo: The Matrix?

Morpheus: Do you want to know what it is? The Matrix is everywhere. It is all around us, even now in this very room. You can see it when you look out of your window, or when you turn on your television. You can feel it when you go to work, when you go to church, when you pay your taxes. It is the world that has been pulled over your eyes to blind you from the truth.

Neo: What truth?

Morpheus: That you are a slave, Neo. Like everyone else, you were born into bondage, born into a prison that you cannot smell or taste or touch, a prison for your mind ... unfortunately, no one can be told what the Matrix is. You have to see it for yourself. This is your last chance. After this there is no turning back. You take the blue pill, the story ends, and you wake up in your bed and believe whatever you want to believe. You take the red pill, you stay in wonderland, and I show you how deep the rabbit hole goes ... remember, all I'm offering is the truth, nothing more ... follow me ... Apoc, are we online?

Apoc: Almost.

Morpheus: Time is always against us. Please, take a seat there.
\end{myquote}

\begin{figure}[htb]
\centering
\includegraphics[width=0.5\linewidth]{fig/read_Matrix-18}
\end{figure}

在第一部中,Morpheus就是我们眼中的神,他无所不知,而且是Neo的老师。后面两集里他的变化也是一个很有意思的讨论话题,到后面再说这个问题。

这里又是长篇大论的对话。这段对话不是太复杂,拣几个有意思的部分来说说。

You can see it when you look out of your window, or when you turn on your television. Morpheus说,你从窗户看出来就能看到Matrix,打开电视也能看到它。窗户外面的Matrix,我们都知道什么意思,因为窗户外面的世界就是Matrix。那电视呢?这就是我想说的另外一个Matrix里很重要的标志物。

一开始,Trinity被Smith的卡车逼到的那个电话亭,后面就是摆了一堆的电视机的商店。

Morpheus和Neo上课时,中间一部大电视。

后来他们被Cypher出卖那次,“出口”电话也是在摆满电视机的地方。

Oracle厅里,也是一部大电视。

那电视代表什么?真正带给我灵感的是Animatrix。当人类发起对机器的进攻时,各种宗教人士在战壕里给战士们祷告,诵经……突然,居然有几个人推着一个电视出来,战壕里的人照样对它膜拜。

宗教在Matrix里代表一种麻醉,束缚人类思想的工具。简单来说,告诉你有上帝,你就不会去想“世界是怎么来的”这样的问题,反正都是上帝造的。宗教是个笼子,困住了人类自由的思想。

电视也是同样的东西,大量的垃圾节目充斥屏幕,人类被电视一次又一次洗脑。

You can feel it when you go to work, when you go to church, when you pay your taxes. Morpheus说,上班,上教堂,缴税时你能感觉到Matrix。这什么意思?为什么这三个方式能让你“感觉”到Matrix?

我们首先得有一个概念,Matrix不是仅仅指那个程序世界,那些代码,它代表的更重要的是对人的mind(思想)的一种禁锢。Matrix让你不敢想,不能想!

Neo上班,被老板骂,听老板的话,老板让他做什么,他就做什么,听从命令。

上教堂,教会的牧师讲道,你虽然不用听从他们的命令,但是你一旦信了教,你和其它信徒“想得就一样了”,再也没有自己的思想。

缴税呢?导演兄弟说过,税是一种每个人都得面对,但是却总是不能负担的东西。就是说,它让你觉得自己的税怎么也交不完。这实际指的是Matrix的界限漫无边际,你很难突破它,所以缴税也能让你“感觉”到Matrix。

Neo选择了红色药丸,这里有我前面说过的“镜子”的摄影手法。

\begin{figure}[htb]
\centering
\includegraphics[width=0.5\linewidth]{fig/read_Matrix-19}
\end{figure}

\myparsep

\begin{myquote}
Neo: You did all this?

Trinity: Uh-huh.

Morpheus: The pill you took is part of a trace program. It's designed to disrupt your input/output carrier signal, so we can pinpoint you.

Neo: What does that mean?

Cypher: It means buckle your seat belt, Dorothy, because Kansas is going to bye-bye.
\end{myquote}

这个红色药丸的作用很简单,它除了是个跟踪程序,更重要的功能是让Neo从真实世界里醒来,从Matrix里拔插头。

Cypher说的这句话来自另外一部超级经典的童话——绿野仙踪(The Wizard of Oz)。Dorothy离开自己在Kansas的家,她的历险过程可以这样概括:她好心肠,喜欢帮助别人,所以这些人也来帮助她,最后她终于回家了。这个童话同样有很多给大人的寓意在里面。

\begin{myquote}
Neo: Did you ...

Morpheus: Have you ever had a dream, Neo, that you were so sure was real? What if you were unable to wake from that dream? How would you know the difference between the dream world and the real world?

Neo: This can't be ...

Morpheus: Be what? Be real?

Trinity: It's going into replication.

Morpheus: Apoc?

Apoc: Still nothing.

Neo: It's cold. It's cold.

Morpheus: Tank, we're going to need a signal soon.

Trinity: We've got fibrillation.

Morpheus: Apoc.

Apoc: Targeting almost there.

Trinity: It's going into arrest.

Apoc: Lock, I've got him.

Morpheus: Now, Tank! Now!
\end{myquote}

这里开始,是Neo吃了红色药丸以后的反应。简单来说,就是他开始发现周围的世界变了,这也是应了“梦由心生”那句话,当你对世界的认识变了的时候,周围的世界就变了。

Neo回头看镜子,镜子本来是两个碎片,居然合在了一起。这里除了表示Neo意识变化,同时也是在说“两个他”己经开始慢慢合并了。

\begin{figure}[htb]
\centering
\includegraphics[width=0.5\linewidth]{fig/read_Matrix-20}
\end{figure}

Neo伸手去摸镜子,居然被“吸入了水银”,这一幕的含义是:一是水银是种比较“寒冷”的东西,代表Neo己经开始感觉到他那个寒冷的胶囊了;二是水银像镜子,同时又在反射这个世界,说明他的本质不是他这个“人”,而是那个“水银”,他只是这个Matrix世界里信息的一个“投影”罢了。

\begin{figure}[htb]
\centering
\includegraphics[width=0.5\linewidth]{fig/read_Matrix-21}
\end{figure}

Trinity说的fibrillation不简单啊,这个词是“纤维性颤动”的意思。在Jean Baudrillard的另一本书America中,关于它的起因,摘录如下:

Like a fibrillation of muscles, striated by the excess of heat and speed, by the excess of things seen or read, of places passed through and forgotten. The defibrillation of the body overloaded with empty signs, functional gestures, the blinding brilliance of the sky, and somnambulistic distances, is a very slow process.

很强吧,导演连这个都放进来了。

\begin{figure}[htb]
\centering
\includegraphics[width=0.5\linewidth]{fig/read_Matrix-22}
\end{figure}

\begin{figure}[htb]
\centering
\includegraphics[width=0.5\linewidth]{fig/read_Matrix-22-1}
\end{figure}

Neo吃了红色的药丸,他就从工厂里醒来。负责拔插头的机器人飞来,它不管Neo的“死活”,从Matrix里醒来的人,在它看来,就是“死了”。它拔了Neo的插头,然后把他扔进下水管道。

这里,我想当初大家看第一部时肯定觉得奇怪,为什么机器要把醒来的人放走?既然章鱼到处杀人,工厂里的机器干嘛不一下戳死Neo?答案就在第二、三部,因为机器就是要让他们去Zion,这是“安排”好的事情。

\myparsep

\begin{myquote}
(Nebuchadnezzar)

Morpheus: Welcome to the real world. We've done it, Trinity. We've found him.

Trinity: I hope you're right.

Morpheus: I don't have to hope. I know it.

Neo: Am I dead?

Morpheus: Far from it.
\end{myquote}

Neo在飞船里醒来,就和Kid—样,迷迷糊糊地听到了Morpheus和Trinity的对话。他问Morpheus“我死了没?”,Morpheus说“还早呢”(你的the One之路刚刚开始)。

\begin{myquote}
Dozer: He still needs a lot of work.

Neo: What are you doing?

Morpheus: Your muscles have atrophied. We're rebuilding them.

Neo: Why do my eyes hurt?

Morpheus: You've never used them before. Rest, Neo. The answers are coming.
\end{myquote}

\begin{figure}[htb]
\centering
\includegraphics[width=0.45\linewidth]{fig/read_Matrix-24}
\includegraphics[width=0.45\linewidth]{fig/read_Matrix-25}
\end{figure}

Morpheus他们用“针灸”来给Neo恢复他萎缩的肌肉(因为从来没用过)。电影里连续的两个镜头很有意思:从Neo的脑袋看下去,我们看到的针都是红色的尾巴;可是从侧面看,我们看到了很多蓝色的尾巴。我觉得应该是在说用脑子看世界和用其它感觉看世界的不同结果,不展开说了。

\begin{figure}[htb]
\centering
\includegraphics[width=0.5\linewidth]{fig/read_Matrix-26}
\end{figure}

Neo醒来摸自己的“接口”,像不像太阳?人类变成了机器的太阳的替代品。

\begin{myquote}
Neo: Morpheus, what's happened to me? What is this place?

Morpheus: More important than what is when.

Neo: When?

Morpheus: You believe it's the year 1999, when in fact it's closer to 2199. I can't tell you exactly what year it is, because we honestly don't know. There's nothing I can say that will explain it for you, Neo. Come with me. See for yourself. This is my ship, the Nebuchadnezzar. It's a hovercraft. This is the main deck. This is the core where we broadcast our pirate signal and hack into the Matrix. Most of my crew you already know. This is Apoc, Switch, and Cypher.
\end{myquote}

Morpheus说到了一个很重要的概念——时间。后面法国人说:“谁有时间呢?”他的意思是你们这些黑客也没有时间,我才有,我才是真正的强者。电影里在传递这个观念:谁有时间谁就更强。Neo能停下子弹,准确地说,他能让时间停下,“拥有”了时间,所以他是强者。

当然,拥有时间,不是像停下子弹那么简单。准确地说,要拥有时间,你起码得知道现在准确的时间。

整个Zion的人,都以为他们是第一代的Matrix,以为现在的时间靠近2199年,可是他们都错了,他们都“没有时间”。更别提Matrix里的人了。

而像Oracle、Architect、Smith、法国人、他老婆,哪怕是他们手下的小爪牙,所有的程序都有时间,因为他们都知道“准确的时间”。

所以电影里的“知道什么时候(when)”==了解Matrix的真相!

Morpheus的船叫Nebuchadnezzar。第二部里Morpheus、Trinity和Keymaker在车上被追杀的时候,他们开的车的车牌是DA203,圣经的Daniel 2:3是:“I have had a dream that troubles me and I want to know what it means.”说这句话的人是个国王,名字就是Nebuchadnezzar。下面这张图也有很多暗示:

\begin{figure}[htb]
\centering
\includegraphics[width=0.5\linewidth]{fig/read_Matrix-27}
\end{figure}

Mark III NO.11,圣经的Mark 3:11是:“Whenever the impure spirits saw him, they fell down before him and cried out, `You are the Son of God.'”

牌子上说飞船是2069年造的,因为这里的飞船是真的,机器也没必要去改这个时间,所以我们大概可以知道,机器和人的战争是在2069年后的某一个时间,比如2079、2089什么的爆发的。这飞船就是那时的人造出来的,上面弹痕累累。机器是在打败人类后,把飞船捡来,修好了,再丢给Zion的。

下面是艾滋病毒的图。你们看出门道了吗?(这是看mop上医盟的朋友发的帖子才知道的)人类就是病毒啊,连飞船都造得和病毒一样。

\begin{figure}[htb]
\centering
\includegraphics[width=0.5\linewidth]{fig/read_Matrix-28}
\end{figure}

\myparsep

\begin{myquote}
Cypher: Hi.

Morpheus: The one's you don't know, Tank, and his big brother, Dozer. The little one behind you is Mouse. You wanted to know what the Matrix is, Neo? Trinity ... try to relax. This will feel a weird.

(Construct)

Morpheus: This is the construct. It's our loading program. We can load anything, from clothing to equipment, weapons, training simulations, anything we need.

Neo: Right now we're inside a computer program?

Morpheus: Is it really so hard to believe? Your clothes are different. The plugs in your arms and head are gone. Your hair is changed. Your appearance now is what we call residual self-image. It is the mental projection of your digital self.

Neo: This ... this isn't real?

Morpheus: What is real? How do you define real? If you're talking about what you can feel, what you can smell, what you can taste and see, then real is simply electrical signals interpreted by your brain. This is the world that you know, the world as it was at the end of the twentieth century. It exists now only as part of a neural-interactive simulation that we call the Matrix. You've been living in a dream world, Neo. This is the world as it exists today ... welcome to the desert of the real. We have only bits and pieces of information, but what we know for certain is that at some point in the early twenty-first century all of mankind was united in celebration. We marveled at our own magnificence as we gave birth to AI.

Neo: AI? you mean artificial intelligence?

Morpheus: A singular consciousness that spawned an entire race of machines. We don't know who struck first, us or them, but we know that it was us that scorched the sky. At the time they were dependent on solar power and it was believed that they would be unable to survive without an energy source as abundant as the sun. Throughout human history, we have been dependent on machines to survive. Fate it seems is not without a sense of irony. The human body generates more bio-electricity than a 120-volt battery and over 25,000 Btu's of body heat. Combined with a form of fusion, the machines have found all the energy they would ever need. There are fields, endless fields, where human beings are no longer born. We are grown. For the longest time I wouldn't believe it. And then I saw the fields with myown eyes, watch them liquefy the dead so they could be fed intravenously to the living. And standing there, facing the pure horrifying precision, I came to realize the obviousness of the truth. What is the Matrix? Control. The Matrix is a computer-generated dream world built to keep us under control, in order to change a human being into this.

Neo: No. I don't believe it. It's not possible.

Morpheus: I didn't say it would be easy, Neo. I just said it would be the truth.

Neo: Stop! Let me out! Let me out! I want out!
\end{myquote}

\begin{figure}[htb]
\centering
\includegraphics[width=0.5\linewidth]{fig/read_Matrix-29}
\end{figure}

呵呵,长篇大论又来了。

Neo拔插头的过程,可以分成肉体上的,和精神上的。那些针灸什么的是肉体的,现在就是精神的,这是他精神的第一课——“what is real”(什么是真实)。

在讨论这个问题之前,先说“residual self-image”(个人影像残余)。Morpheus在说为什么Neo的衣服到了程序世界就换了,头发也变了的时候,Morpheus用了这个RSI,想比较形象地解释它。

我们要从人的婴儿时代说起,人在8个月到18个月大期间,就开始意识到自己是什么模样的,或者说开始知道镜子里的是自己。之前,婴儿根本不知道自己是什么样的,他们脑子里有个人影像,但是却是错误的。

这同时也可以解释Neo吃红色药丸后的那个镜子的场景,因为他原来对于自己的“错误影像”开始被纠正,所以他的“镜子”就变了。

“个人影像残余”实际指的是在这个认识过程中留下的一些没能改变的东西,这也说明接入Matrix,或者接入飞船里的模拟程序construct,就像做梦一样,能挖到一些我们平常想不到的东西。

我们现在来看“什么是真实”。

How do you define real? If you're talking about what you can feel, what you can smell, what you can taste and see, then real is simply electrical signals interpreted by your brain. 这里给出了一个很残酷的现实:我们的“真实”不过是脑子的信号,如果这个信号是假的,我们就分不出真假了。我觉得我们现在在这里谈论真假,很多时候是在拿“假”做参照物。就像讨论Zion的真假,我们说的一个前提是,程序世界是假的。可是我们如果不知道什么是程序世界,怎么知道什么是“真的”?

所以我的理解就是,如果你不知道什么是“假”的,你就不知道什么是“真”了。很遗憾, 我们现在就是这样一种状态,根本分不出真假。

Morpheus接着说到的关于人类和机器的战争、天空、能量的事,还不如Animatrix里讲的详细。我这里简单复述一下:

最早,人就像现在一样,控制机器。20XX年,AI诞生了,机器有了自己的mind。有一天,一个机器觉醒了,开始想要“人权”(生存权),星星之火燎原,最后变成了机器的大革命,它们建立了自己的国家01。在机器要求加入联合国未果后,人类发动了对机器的战争,人类显然不是对手。因为机器的当时的能源主要来自于太阳能,人类就撒下了电磁云,隔断机器和太阳的联系。可是机器却找到了用人的生物能来替代太阳能的方法,它们活了下来。人类被彻底打败,沦为电池。

这里有段Morpheus的话,值得研究:

The human body generates more bio-electricity than a 120-volt battery and over 25,000 Btu's of body heat. Combined with a form of fusion, the machines have found all the energy they would ever need.

除了人类25,000 Btu的体热,他还说到了核能。所以机器是利用了人的能量,但这不是它们的全部能量来源。

Morpheus说,很多人不相信机器能用人做能量,但是如果你亲眼看到无边无际的“人田”,你就彻底醒悟了。

Morpheus和Neo他们坐的沙发还是他们在Matrix里的那个,他们也在看电视。电视只能让你“看到”,真相却需要你“想”才能发现。那个电视的牌子叫“Deepimage”,造型也很特别。

\myparsep

\begin{myquote}
(Nebuchadnezzar)

Trinity: Easy, Neo. Easy.

Neo: Take this thing off me. Take this thing ...

Morpheus: Listen to me ...

Neo: Don't touch me. Stay away from me. I don't want it. I don't believe it. I don't believe it.

Cypher: He's gonna pop.

Morpheus: Breathe, Neo. Just breathe.

Neo: I can't go back, can I?

Morpheus: No. But if you could, would you really want to? I feel I owe you an apology. We have a rule. We never free a mind once it's reached a certain age. It's dangerous. The mind has trouble letting go. I've seen it before and I'm sorry. I did what I did because ... I had to. When the Matrix was first built, there was a man born inside who had the ability to change whatever he wanted, to remake the Matrix as he saw fit. It was he who freed the first of us, taught us the truth. As long as the Matrix exists, the human race will never be free. After he died, the Oracle prophesied his return and that his coming would hail the destruction of the Matrix and the war, bring freedom to our people. That is why there are those of us who have spent our entire lives searching the Matrix looking for him. I did what I did, because I believe that search is over ... get some rest, you're going to need it.

Neo: For what?

Morpheus: Your training.
\end{myquote}

Morpheus说,他们一般不给像Neo这把年纪的人拔插头,因为他们接受不了“真实”。因为年龄在20岁以下的年轻人,世界观往往没有成型,比较容易接受这种“人生巨变”。

Morpheus第一次说到了the One和Oracle的预言,这里是他的原话:

After he died, the Oracle prophesied his return and that his coming would hail the destruction of the Matrix and the war, bring freedom to our people.

翻译:他(第五个the One)死后,Oracle说他还会回来,而且他将带来Matrix的毁灭,结束战争,给我们的人(Zion)带来自由。

Oracle没有骗人,Smith是要毁了Matrix,这是the One(Neo)引起的,战争结束了,Zion的人自由了。

Morpheus可能以为说是Matrix的人自由了,可是他没有想到,其实Zion的人之前也是不“自由”的,所以这里说的freedom是给Zion的人的。

\begin{myquote}
Tank: Morning, did you sleep? You will tonight, I guarantee it. I'm Tank. I'll be your operator.

Neo: You don't ... you don't have any ...

Tank: Holes? Nope. Me and my brother, Dozer, we're both one hundred percent pure, old fashioned, home grown human, born free right here in the real world, a genuine child of Zion.

Neo: Zion?

Tank: If the war was over tomorrow, Zion is where the party would be.

Neo: It's a city?

Tank: The last human city. The only place we have left.

Neo: Where is it?

Tank: Deep underground, near the earth's core where it's still warm. Live long enough and you might even see it. God damn, I ... I got to tell you, I'm fairly excited to see what you're capable of, if Morpheus is right and all ... I'm not supposed to talk about this, but if you are ... a very exciting time. We got a lot to do. We got to get to it ... now, we're supposed to start with these operation programs first, that's a major boring shit. Let's do something more fun. How about combat training.

Neo: Jujitsu? I'm going to learn jujitsu? ... holy shit!

Tank: Hey Mikey, I think he likes it. How about some more?

Neo: Hell yes. Hell yeah.

Morpheus: How is he?

Tank: Ten hours straight. he's a machine.
\end{myquote}

这里是Tank用的键盘:

\begin{figure}[htb]
\centering
\includegraphics[width=0.5\linewidth]{fig/read_Matrix-30}
\end{figure}

这里的图不是很清楚,看电影吧,Ctrl键变成了一个像眼睛一样的东西,Esc键变成了一个圆圈。大家想想什么意思吧,我回头再来说我的想法,都说了就没意思了。

\myparsep

\begin{myquote}
Neo: I know kung fu.

Morpheus: Show me.

(Construct)

Morpheus: This is a sparring program, similar to the programmed reality of the Matrix. It has the same basic rules, rules like gravity. What you must learn is that these rules are not different from the rules of a computer system. Some of them can be bent. Others can be broken. Understand? Then hit me if you can ... good. Adaptation, improvisation. But your weakness is not your technique.

(Nebuchadnezzar)

Mouse: Morpheus is fighting Neo.

(Construct)

Morpheus: How did I beat you?

Neo: You're too fast.

Morpheus: Do you believe that my being stronger or faster has anything to do with my muscles in this place? You think that's air you're breathing now? ... again.

(Nebuchadnezzar)

Mouse: Jesus Christ, he's fast. Take a look at his neural-kinetics, they're way above normal.

(Construct)

Morpheus: What are you waiting for? You're faster than this. Don't think you are, know you are ... come on. Stop trying to hit me and hit me.

(Nebuchadnezzar)

Mouse: I don't believe it.

(Construct)

Neo: I know what you're trying to do.

Morpheus: I'm trying to free your mind, Neo, but I can only show you the door, you're the One that has to walk through it. Tank, load the jump program ... you have to let it all go, Neo, fear, doubt, and disbelief. Free your mind.
\end{myquote}

Morpheus和Neo练功的场景,这里可以说的东西很多。

首先Morpheus说这是个“sparring program”,spar除了说是拳击、打架外,也有辩论、争论的意思。他们之间的交手,你可以看成是“打斗”,但更重要的是他们之间的对话,揭示了很多电影的主题。

这里是他们对战前的图:

\begin{figure}[htb]
\centering
\includegraphics[width=0.5\linewidth]{fig/read_Matrix-31}
\end{figure}

我用黄箭头画出了光线进入的角度,这束光把整间屋子分成两半。Morpheus穿着黑衣服,白色的领子,站在黑影里;Neo穿着白色的衣服,黑色的领子,站在阳光下。整个场景,体现的是一种“平衡”,给人太极图的感觉,屋子里的摆设也是非常对称的。

Morpheus在打斗中,更多的扮演的是Neo的老师的角色。他说:“你需要提高的不是技术。(是你的意识)”“在这个环境里,我是因为肌肉的优势比你快,比你强壮吗?你觉得你真的是在呼吸吗?(你如果相信你不用呼吸,你就不用呼吸了)”

Don't think you are, know you are ... come on. Stop trying to hit me and hit me. 这句话最经典,别“认为”你是谁,而要去“认识”你是谁,意思是你要破除原来一些旧的、错误的对自己的看法。不要老想着打我,想想你为什么不能再快一些,你自然就能打到我了。

这里是房间里的“字画”,这是中文吗?我怎么看不懂这三个字合在一起的意思?

\begin{figure}[htb]
\centering
\includegraphics[width=0.5\linewidth]{fig/read_Matrix-32}
\end{figure}

\begin{myquote}
Neo: Whoa, okie dokie. Free my mind.

(Nebuchadnezzar)

Mouse: So what if he makes it?

Apoc: No one's ever made the first jump.

Mouse: I know, I know. But what if he does?

Apoc: He won't.

Mouse: Come on.

Trinity: Come on.

(Construct)

Neo: All right, no problem. Free my mind. Free my mind. All right.

(Nebuchadnezzar)

Mouse: Wha...what does that mean?

Switch: It doesn't mean anything.

Cypher: Everybody falls the first time. Right, Trin?

(Nebuchadnezzar)

Neo: I thought it wasn't real.

Morpheus: Your mind makes it real.

Neo: If you're killed in the Matrix, you die here?

Morpheus: Your body cannot live without the mind.
\end{myquote}

Neo“跳楼”失败,因为他还不能完全破除对自己“看到”的东西的迷信,没能做到100\%的free his mind。

这里我们知道了一件事:死在Matrix,意味着在真实世界也死了,因为你的mind死了,身体也就“活不成”了。

\begin{myquote}
Cypher: I don't remember you bringing me dinner. There is something about him, isn't there?

Trinity: Don't tell me you're a believer now?

Cypher: I just keep wondering, if Morpheus is so sure, why doesn't he take him to see the Oracle?

Trinity: Morpheus will take him when he's ready.

(Construct)

\begin{figure}[htb]
\centering
\includegraphics[width=0.5\linewidth]{fig/read_Matrix-33}
\end{figure}

Morpheus: The Matrix is a system, Neo. That system is our enemy. But when you're inside, you look around. What do you see? Business men, teachers, lawyers, carpenters. The very minds of the people we are trying to save. But until we do, these people are still a part of that system, and that makes them our enemy. You have to understand, most of these people are not ready to be unplugged. And many of them are so inert, so hopelessly dependent on the system that they will fight to protect it. Were you listening to me, Neo, or were you looking at the woman in the red dress?

Neo: I was ...

Morpheus: Look again. Freeze it.

Neo: This ... this isn't the Matrix?

Morpheus: No. It's another training program designed to teach you one thing. If you are not one of us, you are one of them.

Neo: What are they?

Morpheus: Sentient programs. They can move in and out of any software still hard wired to their system. That means that anyone we haven't unplugged is potentially an agent. Inside the matrix, they are everyone and they are no one. We survived by hiding from them, by running from them. But they are the gatekeepers. They are guarding all the doors. They are holding all the keys, which means that sooner or later, someone is going to have to fight them.

Neo: Someone?

Morpheus: I won't lie to you, Neo. Every single man or woman who has stood their ground, everyone who has fought an agent has died. But where they have failed, you will succeed.

Neo: Why?

Morpheus: I've seen an agent punch through a concrete wall. Men have emptied entire clips at them and hit nothing but air. Yet their strength and their speed are still based in a world that is built on rules. Because of that, they will never be as strong or as fast as you can be.

Neo: What are you trying to tell me, that I can dodge bullets?

Morpheus: No, Neo. I'm trying to tell you that when you're ready, you won't have to.
\end{myquote}

这里训练的是对“干探”的认识。镜头一开始,就是我在讨论颜色的帖子里说到的“红人”和“绿人”。

\begin{figure}[htb]
\centering
\includegraphics[width=0.5\linewidth]{fig/read_Matrix-33-1}
\end{figure}

这个场景,导演拍电影时,使用的全部都是双胞胎,甚至三胞胎,大家可以找找,很有意思。这是因为,Mouse写程序时为了省事,用了“copy paste”的方法,技术不够的程序员在编程时经常用这个方法。导演在用“同样的脸”提醒我们这里是“程序世界”。

They can move in and out of any software still hard wired to their system. 干探可以变成所有接在他们系统的“软件”,这包括了里面的程序,还有接在他们工厂的“人”,人的mind在这里也被形容成了“软件”。

这里说到了“规则”,干探虽然看起来很厉害,但他们还是被“规则”限制,所以Neo肯定能够击败他们,因为the One不被规则限制。

还是那个“镜子”的话题,下面两张图:

\begin{figure}[htb]
\centering
\includegraphics[width=0.5\linewidth]{fig/read_Matrix-34}
\end{figure}

这张图的背景是Morpheus在说:“有人迟早得和他们打。”他的左边眼镜是干探拿着枪对着Neo,右边是没有被枪指的Neo。对应他说的话,Neo迟早会碰上被干探指头的事。再联系我前面说的“左右”,Morpheus的意思是,这个时候,你最好选择不要被指头(右边眼镜)。

\begin{figure}[htb]
\centering
\includegraphics[width=0.5\linewidth]{fig/read_Matrix-35}
\end{figure}

这张图的背景则是Morpheus在说:“你一定会击败干探。”这时右边变成了被枪指头,意思就是到了Neo变成the One的时候,你就可以面对干探,迎战他们,击败他们。

这个场景,导演也非常强调“绿色”,旁边的建筑工地整个用绿色帆布围着,周围很多建筑物也漆了绿边,甚至导演拍这场戏时,为了等叶子变绿,从冬天推迟到了春天,镜头里有些柱子也被故意用绿色的布包了起来。

这个场景里还有红色美女,后面到Mouse那段再来说这个。她挺漂亮的……所以截了个图放这里,大家看着“绿色世界”时,养养眼。

\begin{figure}[htb]
\centering
\includegraphics[width=0.5\linewidth]{fig/read_Matrix-36}
\end{figure}

\myparsep

\begin{myquote}
(Nebuchadnezzar)

Tank: We've got trouble.

Morpheus: Did Zion send word?

Dozer: No, another ship. Shit. Squiddies. We've been in crick.

Neo: Squiddy?

Trinity: A sentinel. A killing machine designed for one thing ...

Dozer: Search and destroy.

Morpheus: Set her down right over there ... how're we doing, Tank?

Tank: Power off line. EMP armed and ready.

Neo: EMP?

Trinity: Electromagnetic pulse. Disables any electrical system in the blast radius. It's the only weapon we have against the machines.

Neo: Where are we?

Trinity: There old service and waste systems.

Neo: Sewers?

Trinity: There used to be cities that spanned hundreds of miles. Now these sewers are all that's left of them.

Morpheus: Quiet.
\end{myquote}

Morpheus戴起了小帽子,除了因为飞船关了电源,暖气肯定也关了,挺冷的以外,他怕他的光头“发亮”,被章鱼看到。导演在这里偷偷幽默了一把。

\begin{myquote}
Cypher: Whoa, Neo. You scared the bejeezus out of me.

Neo: Sorry.

Cypher: It's okay.

Neo: Is that ...

Cypher: The Matrix? yeah.

Neo: Do you always look at it encoded?

Cypher: Well you have to. The image translators work for the construct program, but there's way too much information to decode from the Matrix. You get used to it. I ... I don't even see the code. All I see is blonde, brunette, red-head. Hey, you a ... want a drink?

Neo: Sure.

Cypher: You know, I know what you're thinking, because right now I'm thinking the same thing. Actually, I've been thinking it ever since I got here. Why, oh, why didn't I take the blue pill? ... good shit, huh. Dozer makes it. It's good for two things, degreasing engines and killing brain cells. So, can I ask you something? Did he tell you why he did it, why you're here? Jesus. What a mind job. So you're here to save the world. What do you say to something like that? A little piece of advice. You see an agent, you do what we do. Run. Run your ass off.

Neo: Thanks for the drink.

Cypher: Sweet dreams.
\end{myquote}

Neo出现在Cypher后面,吓坏了他,因为Cypher应该是在安排和Smith的会面,他很快把屏幕都关了。

他给Neo的建议是“看到干探就逃跑”,Neo后来说“谢谢你的饮料”(没谢他的建议)。

Cypher说“sweet dreams”也是有意思的,等到第二部Morpheus在Zion说这句话时,我一起解释。

\begin{myquote}
(Restaurant)

Agent Smith: Do we have a deal, Mr.~Reagan?

Cypher: You know, I know this steak doesn't exist. I know that when I put it in my mouth, the Matrix is telling my brain that it is juicy and delicious. After nine years, you know what I realize? Ignorance is bliss.

Agent Smith: Then we have a deal?

Cypher: I don't want to remember nothing. Nothing. You understand? And I want to be rich. You know, someone important, like an actor.

Agent Smith: Whatever you want, Mr.~Reagan?

Cypher: Okay. I get my body back into a power plant, you insert me into the Matrix, I'll get you what you want.

Agent Smith: Access codes to the Zion mainframe.

Cypher: No, I told you, I don't know them. I can get you the man who does.

Agent Smith: Morpheus.
\end{myquote}

Smith和Cypher在牛排馆会面,我们知道Cypher的原名居然是里根,而且里根现在也病得“什么都不记得了”,导演在讽刺这个美国前总统。

Cypher在这里说的话很能说明一些问题:“我知道牛排不存在……九年后,你知道我明白了什么?无知真好啊。”就像Morpheus说的:“欢迎来到真实的荒漠。”真实就是这么可怕。无知不好吗,人生苦短,何必这么折腾自己呢?

自己问自己一个问题,真实真的那么重要吗?特别是当你无法分辨真实和不真实时,在这个世界,我们需要的到底是什么呢?

\begin{myquote}
(Nebuchadnezzar)

Tank: Here you go, buddy. Breakfast of champions.

Mouse: If you close your eyes it almost feels like you're eating runny eggs.

Apoc: Yeah, or a bowl of snot.

Mouse: Do you know what it really reminds me of? Tasty wheat. Did you ever eat tasty wheat?

Switch: No, but technically, neither did you.

Mouse: That's exactly my point. Exactly. Because you have to wonder now. How did the machines know what tasty wheat tasted like. huh? Maybe they got it wrong. Maybe what I think tasty wheat tasted like actually tasted like oatmeal or tuna fish. That makes you wonder about a lot of things. You take chicken for example, maybe they couldn't figure out what to make chicken taste like, which is why chicken tastes like everything. Maybe couldn't figure out ...

Apoc: Shut up, Mouse.

Dozer: It's a single cell protein combined with synthetic aminos, vitamins, and minerals. Everything the body needs.

Mouse: It doesn't have everything the body needs. So I understand that you've run through the agent training program. You know, I wrote that program.

Apoc: Here it comes.

Mouse: So what did you think of her?

Neo: Of who?

Mouse: The woman in the red dress? I designed her. She, um ... well she doesn't talk very much, but ... but if you'd like to meet her, I can arrange a much more personalized meeting.

Switch: Digital pimp, hard at work.

Mouse: Pay no attention to these hypocrites, Neo. To deny our own impulses is to deny the very thing that makes us human.

Morpheus: Dozer, when you're done, bring the ship up to broadcast depth. We're going in. Taking Neo to see her.

Neo: See who?

Tank: The Oracle ... everyone please observe the fasten seat belt and no smoking signs have been turned on. Sit back and enjoy your ride.
\end{myquote}

看完了Cypher吃的牛排,我们看到了在飞船上的人在吃“战士早餐”。对比一下,你应该能明白两者的关系了吧。电影很喜欢使用对比来揭示主题。

\begin{figure}[htb]
\centering
\includegraphics[width=0.45\linewidth]{fig/read_Matrix-37}
\includegraphics[width=0.45\linewidth]{fig/read_Matrix-38}
\end{figure}

这段对话非常重要,讨论了“真实”和“人类情感”两个问题。

\begin{figure}[htb]
\centering
\includegraphics[width=0.5\linewidth]{fig/read_Matrix-39}
\end{figure}

Mouse针对“tasty wheat”(美味麦片)对Neo提问,机器给我们设计的“美味麦片”只不过是个信号,我们习惯成自然地接受了,因为这个世界的所有事物都是这样。我们从来没有一个“标准”或者“真理”来限定它,对这个事物的认识都是不断变化的。

机器如果弄错了“美味麦片”的味道,同样道理,人也可能弄错了对一个事物的认识。比如时间,也许我们对它的认识完全错了,但是我们“习惯”了去接受这个错误的认识。

这里,习惯比真实更重要。这也是为什么99\%的人宁可接着做梦,因为习惯了,不愿去打破它。

Mouse接着谈到了那个“红衣女郎”,她代表人类的情感、冲动、非理性的行为(机器眼中),也代表整个Matrix的特点:就算99\%的Matrix是个完美的程序世界,它也总是会被1\%的人类情感因素所影响,或者说1\%的因素发挥了100\%的作用。

To deny our own impulses is to deny the very thing that makes us human. 否认我们的情感冲动,等于否认我们人类的最独特之处。

Mouse这话很经典,解释了人和机器一个很重要的不同点,也解释了为什么机器无法设计出完美的Matrix。人类的这种“不可预计”的行为,机器没法用程序来实现。

Morpheus说去“广播层”,然后去找Oracle,这说明Matrix是地表附近的一个无线网络。

\myparsep

\begin{myquote}
(Lafayette hotel)

Morpheus: We're in ... we'll be back in an hour.
\end{myquote}

\begin{figure}[htb]
\centering
\includegraphics[width=0.5\linewidth]{fig/read_Matrix-40}
\end{figure}

他们回Matrix,也是Neo拔插头后第一次回Matrix。这里他们使用的出口,门口写着“hotel delivery”,刨去字面的意思,hotel = hot tel(热线),所以这个hotel delivery就是黑客们的专用传送通道。旁边还有一个告示牌,写着keep clear(请保持空旷),一语双关,clear也有“清晰”的意思,这里说的是,保持他们的通讯线路正常工作。

\begin{figure}[htb]
\centering
\includegraphics[width=0.5\linewidth]{fig/read_Matrix-41}
\end{figure}

\begin{myquote}
(Car)

\begin{figure}[htb]
\centering
\includegraphics[width=0.5\linewidth]{fig/read_Matrix-42}
\end{figure}

Morpheus: Unbelievable, isn't it?

Neo: God.

Trinity: What?

Neo: I used to eat there. Really good noodles. I have these memories from my life. None of them happened. What does that mean?

Trinity: That the Matrix cannot tell you who you are.

Neo: And the Oracle can?

Trinity: That's different.

Neo: Did you go to her?

Trinity: Yes.

Neo: What did she tell you?

Trinity: She told me ...

Neo: What?

Morpheus: We're here. Neo, come with me.
\end{myquote}

这里的场景在第三部同样出现了。

Neo说:“我从我的生活中得到了这些回忆,它们却都没有真的发生过,这说明了什么?”Trinity 说:“Matrix不会告诉你,你是谁。”这里说到了一个很重要的概念:人的记忆可能是错的,就像电影《记忆碎片》(Memento)说的那样。

\begin{myquote}
(Apartment building)

Neo: So is this the same Oracle that made the prophecy?

Morpheus: Yes. She's very old. She's been with us since the beginning.

Neo: The beginning ...?

Morpheus: Of the resistance.

Neo: And she knows what, everything?

Morpheus: She would say she knows enough.

Neo: And she's never wrong.

Morpheus: Try not to think of it in terms of right and wrong. She is a guide. She can help you to find the path.

Neo: She helped you?

Morpheus: Yes.

Neo: What did she tell you?

Morpheus: That I would find the One ... I told you I can only show you the door. You have to walk through it.
\end{myquote}

从进Oracle的门,到Neo抓着饼干出来,这就是整个第一部电影的精髓,或者说最杀脑细胞的地方。电影里Oracle说话的地方,其实比Architect说的更难理解,如果真能看懂她讲的话,你就可以说:“Matrix,我看懂了。”

这两天进度比较慢,除了前面那个部分我之前就不是很熟悉,多看了两三遍外,还有就是因为Oracle公寓的这一幕,里面的东西太多了,完全可以单独写篇文章来讨论。

\begin{myquote}
(Oracle's place)

Priestess: Hello, Neo. You' re right on time ... make yourself at home, Morpheus. Neo, come with me ... these are the other potentials. You can wait here.
\end{myquote}

Neo被带进来后,没有被马上领去见Oracle,当然不是因为Oracle忙。从那个抱小孩的大妈出来,Oracle就没有其它客人了。而且我们都知道,Neo比那些都重要得多,那Oracle为什么不马上见他?

先从Oracle屋里的那堆小孩说起,那堆小孩被称为potentials(有潜能者),他们看Neo的眼神,简直像是Neo在读程序代码。

他们是Oracle培养的接班人,就像Sati,他们学习的,除了对这个世界的理解,更重要的是对类似Neo这样的the One的“指引”,还有对未来的“预测”。

他们的作用,我曾经有过这样的看法,抛砖引玉,觉得挺有意思,就拿出来给大家开拓一下思路:他们不仅可以看成是Oracle的接班人,还能看成是Oracle用来帮助她实现Matrix稳定的各种“程序”。因为一个Matrix的稳定,未必只是靠一个“the One”来实现的,也许她还设计了其它的“the Two”之类的来帮助系统其它方面的升级。

Neo先看到了那两个玩“空中积木”的小女孩。下面是空中积木的图。我没有想到完全的解释,日后再补齐这个部分的分析,大家也可以来想想,很有意思。那些字符和图形分成三种:红色、蓝色(不带框)、蓝色(带框)。我敢肯定的是,它们代表对未来的“预测”,或者说是整个Matrix电影故事的“投影”。

\begin{figure}[htb]
\centering
\includegraphics[width=0.5\linewidth]{fig/read_Matrix-43}
\end{figure}

\begin{myquote}
Spoon boy: Do not try and bend the spoon. That's impossible. Instead, only try to realize the truth.

Neo: What truth?

Spoon boy: There is no spoon.

Neo: There is no spoon?

Spoon boy: Then you'll see that it is not the spoon that bends. It is only yourself.

Priestess: The Oracle will see you now.
\end{myquote}

\begin{figure}[htb]
\centering
\includegraphics[width=0.5\linewidth]{fig/read_Matrix-45}
\end{figure}

执勺少年来啦~

之前的文章很详细地讨论过这个问题,这里简单说一下:there is no spoon = there is no rule。不要被你眼前看到的规则所限,被你眼前的事物所迷惑,你在规则里不能打破规则。你要“看自己”,而不是“看规则”,这样你才能认识到“真相”,才能拧弯勺子。

这个外国小娃穿上袈裟扮和尚,也是在点明他的作用——“点化Neo”。

就是在Neo明白这事之后,Oracle要见他了。

\begin{myquote}
Oracle: I know you're Neo. Be right with you.

Neo: You're the Oracle?

Oracle: Bingo. Not quite what you were expecting, right? Almost done. Smell good, don't they?

Neo: Yeah.

Oracle: I'd ask you to sit down, but your not going to anyway. And don't worry about the vase.

Neo: What vase?

Oracle: That vase.

Neo: I'm sorry.

Oracle: I said don't worry about it. I'll get one of my students to fix it.

Neo: How did you know?

Oracle: What's really going to bake your noodle later on is, would you still have broken it if I hadn't said anything? You're cuter than I expected. No wonder she likes you.

Neo: Who?

Oracle: Not too bright, though. You know why Morpheus brought you to see me?

Neo: I think so.

Oracle: So, what do you think? You think you're the One?

Neo: I don't know.

Oracle: You know what that means? It's Latin, means ``know thyself''. I'm going to let you in on a little secret. Being the One is just like being in love. No one can tell you you're in love. You just know it, through and through, balls to bones. Well, I better have a look at you. Open your mouth. Say ahhh.

Neo: Ahhh.

Oracle: Okay. Now I'm supposed to say, umm, that's interesting, but ... then you say ...

Neo: But what?

Oracle: But you already know what I'm going to tell you.

Neo: I'm not the One.

Oracle: Sorry kiddo. You got the gift, but it looks like you're waiting for something.

Neo: What?

Oracle: Your next life maybe, who knows? That's the way these things go. What's funny?

Neo: Morpheus. He ... he almost had me convinced.

Oracle: I know. Poor Morpheus. Without him we're lost.

Neo: What do you mean, without him?

Oracle: Are you sure you want to hear this? Morpheus believes in you, Neo. And no one, not you, not even me can convince him otherwise. He believes it so blindly that he's going to sacrifice his life to save yours.

Neo: What?

Oracle: You're going to have to make a choice. On the one hand you'll have Morpheus' life and on the other hand you'll have your own. One of you is going to die. Which one will be up to you. I'm sorry, kiddo, I really am. You have a good soul, and I hate giving good people bad news. Oh, don't worry about it. As soon as you step outside that door, you'll start feeling better. You'll remember you don't believe in any of this fate crap. You're in control of your own life, remember? Here, take a ... I promise, by the time you're done eating it, you'll feel right as rain.

Morpheus: What was said was for you and for you alone.
\end{myquote}

\begin{figure}[htb]
\centering
\includegraphics[width=0.5\linewidth]{fig/read_Matrix-46}
\end{figure}

Neo就像第三部的Oracle说的一样,像只“六月虫”一样钻进Oracle的厨房,一脸茫然,不知所措。他看到的Oracle更让他吃惊:一个在厨房弄饼干的大妈……

电影的截图不是很清楚,Oracle的冰箱上的那些“字母”和“数字”也是有意义的,太多了,等第三部的“冰箱”再出现时,一起讨论。

\begin{figure}[htb]
\centering
\includegraphics[width=0.5\linewidth]{fig/read_Matrix-47}
\end{figure}

Oracle为了让Neo死心塌地听她的,开始用“骗”的了。大家看上面Neo进来的那个图,花瓶和下面的架子就摆在门边,而且Neo—进来,Oracle就让他站那别动。如果你仔细看电影,Neo一进来,其实己经在“磕磕碰碰”花瓶了。

再看下面的图,我从Zion archive里找来的(第三部的Oracle的厨房),大家看到那个花瓶和架子,己经被搬得离门很远了,说明那个花瓶在第一部是故意放那儿等着“破”的。Smith在第三部摔饼干的情节,实际是在对应这里的“花瓶情节”,让观众理解Oracle下面说的那句话的意思。

\begin{figure}[htb]
\centering
\includegraphics[width=0.5\linewidth]{fig/read_Matrix-48}
\end{figure}

花瓶不过是个代码,Neo进来时,己经和花瓶“挨上”了,Oracle控制Neo不行,控制一段代码——花瓶却是轻而易举,花瓶就那么碎了,Oracle就被看成了“先知”。

这里很有趣的是,Oracle和Neo说:“你要搞清楚的是,如果我什么都没说,花瓶还会碎吗?”我想,作为观众的我们,看了第一部,肯定觉得她的意思是那个花瓶注定要碎了。但是,如果你理解了Oracle“预见”未来的能力,你就明白她的意思是:“如果我什么都没说,花瓶根本不会碎!”

Oracle为什么要Neo相信她?很简单,因为她要让Neo走她安排的the One的路。

她也说到了要让她的“孩子”来修花瓶,第三部我们看到“修好的花瓶”根本就是和新的一样,不是补出来的,说明她的那些“孩子”也有修改Matrix代码的能力,她在培养和自己功能差不多的“孩子”。

接着Oracle说到了:“我知道她为什么喜欢你了。”她指的是Trinity,关于Oracle的“Trinity会爱上the One”的预言,其实应该反过来理解,那是一个规则:“只有Trinity爱的人才可能是the One。”注意,我说的是“才可能”,这只是一个必要条件,而不是充分条件。

Oracle为什么要设定这个规则?我们看完三部回头来想,之所以Neo能完成这么多“不可能”的任务,他和Trinity的爱起了决定性的作用。

爱,人的冲动,疯狂、没有理性的行为,是机器的程序不能预计和理解的。Oracle就是知道这个,所以让这一代的the One经历“爱”,因为她要“革命”,要“改变”,需要一个特殊的the One。

下面这个情节就有趣了,Oracle像医生一样要Neo张嘴,看他的喉咙,还给他看面相。她在做什么呢?

\begin{figure}[htb]
\centering
\includegraphics[width=0.5\linewidth]{fig/read_Matrix-49}
\end{figure}

Oracle叫Neo说的是“ahhh”,就是A--H。大家可能注意到了,Oracle的冰箱上面有两本电话簿,city phone的。我在Zion archive里找到了它的近照:

\begin{figure}[htb]
\centering
\includegraphics[width=0.5\linewidth]{fig/read_Matrix-50}
\end{figure}

看出来了吗?那本是L--Z。A--H,L--Z,中间少了IJK,IJK = I'm Just Kidding(我逗你玩呢)。这里是Oracle逗Neo玩,说白了,是导演和我们开玩笑呢。

Oracle接着说Neo不是the One,然后她说Neo有天分,但是在等什么,“也许是来生?”她在预言Neo后来死而复生的事了。

Oracle说:“可怜的Morpheus,没有他我们就输了。”这句话,非常非常有趣。没看到后面的情节,你可能觉得她说的是:没有Morpheus,找不到the One,“我们”就输了。

可是接着,Oracle说了Neo要选择自己死还是Morpheus死。回头来看上面这句,我们在什么情况下“失去Morpheus?”Neo如果不去救Morpheus,他就不会变成the One,所以……这个lost,我觉得解释成“不知道该怎么办”比较好。

所以Oracle的意思是,现在是对你的考验,如果你选择不救Morpheus,你就不是the One,我们还得接着找the One。

Oracle最后给了Neo—个饼,看下面的图,这里是6个;到了第三部,她做的是7个。虽然既可以说是第六代的the One,或者第六代的Matrix,但我倾向是前者,相对Architect来说,Oracle更多地在负责设计the One的道路。

\begin{figure}[htb]
\centering
\includegraphics[width=0.5\linewidth]{fig/read_Matrix-51}
\end{figure}

之前有朋友问过,在原始剧本里,Oracle呆的地方叫Temple(神殿),而且是在Zion的电脑主机里的程序。我想电影这样改,是非常成功的,看完第一部,我们连Oracle是不是程序都不知道。到了第二部Neo问了,她承认了,我们才确认。

这也是电影一个主要的特点,或者说是主题思想,我一直重复在说的:“不要相信你看到的。”程序不是代码,程序看起来就是人,而且比人还像“人”。

我非常喜欢Oracle这个角色,她完全就是个邻居大妈的形象,和蔼可亲,你怎么能把她和控制Matrix的机器程序联系起来?电影用这个手法,来说明它的主题:“外观往往具有欺骗性。”这个世界不是“看到的”那么简单,“想到的”才比较接近真相。

\myparsep

\begin{myquote}
(Nebuchadnezzar)

Tank: They're on their way ... what is that?

(Lafayette hotel)

\begin{figure}[htb]
\centering
\includegraphics[width=0.5\linewidth]{fig/read_Matrix-53}
\end{figure}

Neo: Whoa, deja vu.

Trinity: What did you just say?

Neo: Nothing, I just had a little deja vu.

Trinity: What did you see?

Cypher: What happened?

Neo: A black cat went past us, and then another that looked just like it.

Trinity: How much like it? Was it the same cat?

Neo: Might have been. I'm not sure.

Morpheus: Switch, Apoc.

Neo: What is it?

Trinity: Deja vu is usually a glitch in the Matrix. It happens when they change something.
\end{myquote}

开始了,电影的高潮部分终于开始了。从这里到电影结束,故事的节奏很快,等你好不容易觉得可以喘口气了,Neo己经打完电话,飞没了。

\begin{figure}[htb]
\centering
\includegraphics[width=0.5\linewidth]{fig/read_Matrix-52}
\end{figure}

这时的Neo是他们那堆人里唯一没有戴墨镜的,说明的是,他还不是很懂得保护自己。

Deja vu,我觉得它完全可以取代mop里“火星”这个词,任何一个半年前出现过的帖子再被人zt回来,难道不是一种deja vu?

在电影里,或者说在Matrix里,deja vu代表机器修改了部分代码。第三部里同样的“黑猫”场景出现,代表机器修改了Matrix的代码,新一代的Matrix开始运行。

Morpheus在这里说:“Switch, Apoc.”表面上是叫两个人的名字,但Switch = 切换,Apoc = 启示录,代表的意思是“上帝”(机器)安排的“耶稣”(the One)之路开始了。

这里的黑猫脚下的地板是黑白相间的,机器和黑客双方交手的地方,都用了这个图案的地板。我在之前分析“颜色”的文章里讨论过这些,简单重复一下:黑白相间代表的是机器的“白道”和黑客们的“黑道”交手的战场,不是因为黑客有个“黑”就说他们是“黑道”,而是因为机器控制了能源,就是控制了“光”,所以没有“光”的“自由的人”就是处在黑暗中。

那些警察上来的场景里,我们看到的地板也是黑白相间。

\begin{figure}[htb]
\centering
\includegraphics[width=0.5\linewidth]{fig/read_Matrix-54}
\end{figure}

\begin{myquote}
(Nebuchadnezzar)

Tank: Oh, my God.

(Lafayette hotel)

Morpheus: Let's go.

(Nebuchadnezzar)

Tank: They cut the hard line. It's a trap. Get out.

(Lafayette hotel)

Mouse: Oh, no. Oh, no.

Cypher: That's what they changed. We're trapped. There's no way out.

Morpheus: Be calm. Give me your phone.

Trinity: They'll be able to track it.

Morpheus: We have no choice.

(Cellular)

Tank: Operator.

Morpheus: Tank, find a structural drawing of this building. Find it fast.

Tank: Got it.

Morpheus: I need the main wet wall.
\end{myquote}

Morpheus说要从“wet wall”走,这个wet很有意思。每次他们要跑路,都和水有关,比如Trinity 开始要去的电话亭在“Wells and Lake”。这样的例子很多,电影是在借用“水”的“救命”这一象征意义。

\begin{myquote}
(Lafayette hotel)

Agent Smith: Eighth floor.

Agent Brown: Eighth floor.

Morpheus: Switch, straight ahead.

Apoc: Neo, I hope the Oracle gave you some good news.

(Cellular)

Tank: Another left. That's it.

Morpheus: Good.
\end{myquote}

这里有个东西值得提一下:Tank使用的键盘。

\begin{figure}[htb]
\centering
\includegraphics[width=0.5\linewidth]{fig/read_Matrix-55}
\end{figure}

我们可以把下面那个键盘的那排白色图案和数字键0--9对应起来。因为这里的分析是个人猜测,大家不要太当真,就当作开拓思路,我也只说两个:“)”和“(”。

“)”顾名思义,代表一个表达式的结束。对应的数字是4,电影里4代表的是终结、死亡。(同中文的“死”?)

“(”对应“)”来说,自然代表的是“开始”,它对应的数字是“9”。

后来Neo进Matrix去救Morpheus时,他让Tank载入武器,Tank按的是空格键旁边的“)”“(”,49,在 ASCII码中对应的是1,代表的是让Neo变成the One,同时也有“先死后生”的意思。

\begin{figure}[htb]
\centering
\includegraphics[width=0.5\linewidth]{fig/read_Matrix-56}
\end{figure}

上面的图是Morpheus他们最后进那个洗手间之前的画面,Tank指示说:“Another left. That's it.”Morpheus他们是走了“左边”的门去的那个洗手间,这也是电影里“左右”的再次体现,“左门”从来都是死路,要活就一定要走“右门”。

\begin{myquote}
(Lafayette hotel)

Agent Brown: Where are they?

Police: They're in the walls. They're in the walls.

Cypher: It's an agent.

Trinity: Morpheus.

Morpheus: You must get Neo out. He's all that matters.

Neo: No, no, Morpheus. Don't.

Morpheus: Trinity, go.

Trinity: Go.

Neo: We can't leave him.

Trinity: We have to ... Cypher, come on.

Agent Smith: The great Morpheus. We meet at last.

Morpheus: And you are?

Agent Smith: Smith. Agent Smith.

Morpheus: You all look the same to me.

Agent Smith: Take him.
\end{myquote}

这里发生的场景感觉好像很老套,但是在现在这个时候回头来看,我觉得自己有了很多新的理解。Smith一爪子抓向Neo,不是碰巧,而是“安排”好的。他不会去抓Morpheus,不会去抓Trinity,只会抓Neo,因为这是Oracle安排好的,Smith根本就是她写好的程序,在执行命令,完成“目的”而已。

现在,解释Morpheus的名字的时机也成熟了。Morpheus在古希腊传说里是管梦的,代表人类的梦想,电影里他代表的是Zion里的人渴望自由的“梦想”。他坚信the One的预言,其实这不仅是人类的唯一梦想,也是人类的唯一希望。

Oracle设计的the One道路也实在够狠,Neo你选择救Morpheus(实现人类的梦想),还是救你自己,这个选择不仅仅是两条人命那么简单。

这里的地板也是黑白相间的地砖。

\begin{myquote}
(Nebuchadnezzar)

Tank: No.

(Phone)

\begin{figure}[htb]
\centering
\includegraphics[width=0.5\linewidth]{fig/read_Matrix-57}
\end{figure}

Tank: Operator.

Cypher: Yeah, I need an exit fast.

Tank: Cypher?

Cypher: Yeah, there was an accident. God damn car accident. All of a sudden, boom. Somebody up there still likes me.

Tank: Gotcha.

Cypher: Get me out of here fast.

Tank: Intersection of Franklin and Erie, an old TV repair shop.

Cypher: Right.
\end{myquote}

Cypher打电话的背景是辆着火翻车的警车,他说:“God damn car accident.”之前说了很多次了,God damn代表的意思就是机器安排好的。

Somebody up there still likes me. 这句话经常拿来表达“老天爷帮忙”的意思,这里自然是在说“God”(机器)帮他。

\begin{figure}[htb]
\centering
\includegraphics[width=0.5\linewidth]{fig/read_Matrix-58}
\end{figure}

Trinity他们钻出来的背景,那个车还特地用绿色的布蒙着,绿色的使用是在说明“机器安排”发挥的重要作用。

\begin{myquote}
(Cellular)

Trinity: Tank, it's me.

Neo: Is Morpheus alive?

Cypher: Is Morpheus alive, Tank?

Tank: They're moving him. I don't know where to yet.

Trinity: He's alive. We need an exit.

Tank: You're not far from Cypher.

Trinity: Cypher?

Tank: I know. He's at Franklin and Erie.

Trinity: Got it.

(Nebuchadnezzar)

Tank: Got him.

Cypher: Where are they?

Tank: Making the call.

Cypher: Good.

(Repair shop)

Trinity: You first, Neo.

(Nebuchadnezzar)

Cypher: Shoot.

Dozer: No!

(Repair shop)

Neo: I don't know, it just went dead.
\end{myquote}

他们又到了一个修电视机的店,出口在这里。电影一开始,Trinity被撞的那个地方,电话亭后面也是一个电视机店。

这里我想补充一下之前说的“电视”的作用,刚刚想到的。电视在英文里的全称是television,tele代表的意思是“远程”,vision可以说是“视野”的意思,所以连起来代表的意思是“远见”、“深刻的理解”。电影里的电视机,准确地说,代表一种“深刻的思想理解”、“对事物的准确认识”。

\begin{myquote}
(Cellular)

Cypher: Hello, Trinity.

Trinity: Cypher? Where's Tank?

Cypher: You know, for a long time, I thought I was in love with you. I used to dream about you. You're a beautiful woman, Trinity. Too bad things had to turn out this way.

Trinity: You killed them.

Apoc: What?

Switch: Oh, God.

Cypher: I'm tired, Trinity. I'm tired of this war. I'm tired of fighting. I'm tired of this ship, being cold, eating the same God damn goop everyday. But most of all, I'm tired of that jack-off and all of his bullshit surprise ass-hole. I bet you never saw this coming, did you? God, I wish I could be there, when they break you. I wish I could walk in just when it happens. So right then, you'd know it was me.

Trinity: You gave him Morpheus.

Cypher: He lied to us, Trinity. He tricked us. If you'da told us the truth, we woulda told you to shove that red pill right up your ass.

Trinity: That's not true, Cypher, he set us free.

Cypher: Free? you call this free? All I do is what he tells me to do. If I had to choose between that and the Matrix, I choose the Matrix.

Trinity: The Matrix isn't real.

Cypher: I disagree, Trinity. I think the Matrix can be more real than this world. All I do is pull the plug here. But there, you have to watch Apoc die.
\end{myquote}

Cypher的这段对话太精彩了,如果你只看到了一个吃醋而抓狂的小男人,你就错过了电影的精髓。

Cypher和Trinity的讨论,代表现在我们对于“真实”的理解。真实到底是“真实的感觉”,还是“虚拟的相反”呢?

你选择真实,是在选择“感觉”,还是在选择“反抗虚拟”?

其实大多数人,看重的不过是种“感觉”,被不被控制不是那么重要。就像暴政为什么未必会被推翻,只要它够“暴”,人还是宁可选择低头,因为大多数人不在乎是否被“控制”。

\begin{myquote}
(Repair shop)

Apoc: Trinity.

Switch: No!
\end{myquote}

\begin{figure}[htb]
\centering
\includegraphics[width=0.5\linewidth]{fig/read_Matrix-59}
\end{figure}

这个镜头,我们看到修电视的店的地板也是黑白相间的砖。

\begin{myquote}
(Cellular)

Cypher: Welcome to the real world, huh, baby.

Trinity: But you're out, Cypher. You can't go back.

Cypher: Oh, no. That's what you think. They're going to re-insert my body. I go back to sleep, and when I wake up, I won't remember a God damn thing. By the way, if you have anything terribly important to say to Switch, I suggest you say it now.

Trinity: No, please don't.
\end{myquote}

Cypher说:“If you have anything terribly important to say to Switch, I suggest you say it now.”(如果你有什么重要的事情要和Switch说的话,我建议你现在说。)这里的Switch字面意思指的是那个白发MM,可是联系这个词的意思——“转换”,我们要注意Trinity下面说的话,肯定有句话是扭转局势的。

\begin{myquote}
(Repair shop)

Switch: Not like this. Not like this.

(Cellular)

Trinity: God damn you, Cypher.

Cypher: Don't hate me, Trinity. I'm just the messenger, and right now I'm going to prove it to you. If Morpheus was right, then there's no way I can pull this plug. I mean if Neo's the One, then there'd have to be some kind of a miracle to stop me, right? I mean, how can he be the One if he's dead? You never did answer me before. If you bought into Morpheus' bullshit --- come on --- all I want is a little yes or no. Look into his eyes, those big pretty eyes. Tell me. Yes or no.

Trinity: Yes.

Cypher: No.
\end{myquote}

以上的对话里“God damn”出现了很多次,意思还是一样的。

Cypher说自己是“信使(messenger),他说他要拔Neo的插头,如果预言是真的,他是没法拔那个插头的,结果是他没拔成。这里Cypher的的确确就是一个信使,告诉我们Neo就是the One。

他追问Trinity的是,Neo是否是the One,yes还是no。Trinity的那个yes,就是我上面那段提到的非常重要的话,因为只有Trinity“爱的”才可能是the One。Trinity发现自己爱Neo,所以她还是相信Neo是the One,这就是那句“能扭转乾坤”的话。当然,这不是机器安排的,只是导演在说the One道路中最重要的一个因素——Trinity的爱。

有朋友问过我,为什么机器、或者说Oracle能知道Tank打不死,爬起来干掉Cypher,这难道也是机器安排的吗?

当然不是,回头看Morpheus一直在说的:“Oracle给你说的只是说给你听的。”所以就算Trinity 发现自己爱Neo,Neo满足了这个条件,但是他被Cypher拔插头了,死了,没有其它人知道这个事。Morpheus现在实际是被机器保护起来了,Neo、Trinity他们失败了,死了,Morpheus还活着,他可以继续自己的“梦想”,继续寻找the One。因为Oracle告诉他的是“你能找到the One”,而从来没有告诉他“Neo就是the One”,也没有提过任何有关Trinity的事情。

\myparsep

\begin{myquote}
(Nebuchadnezzar)

Cypher: I don't believe it.

Tank: Believe it or not, you piece of shit. You're still gonna burn.

(Repair shop)

Neo: You first.

(Nebuchadnezzar)

Trinity: You're hurt.

Tank: I'll be all right.

Trinity: Dozer?

(Office)

Agent Smith: Have you ever stood and stared at it, marveled at its beauty, its genius? Billions of people just living out their lives, oblivious. Did you know that the first Matrix was designed to be a perfect human world, where none suffered, where everyone would be happy? It was a disaster. No one would accept the program. Entire crops were lost. Some believed that we lacked the programming language to describe your perfect world. But I believe that as a species, human beings define their reality through misery and suffering. The perfect world would dream that your primitive cerebrum kept trying to wake up from, which is why the Matrix was re-designed to this, the peak of your civilization. I say your civilization, because as soon as we started thinking for you, it really became our civilization, which is of course what this is all about. Evolution, Morpheus, evolution, like the dinosaur. Look out that window. You had your time. The future is our world, Morpheus. The future is our time.

Agent Brown: There could be a problem.
\end{myquote}

\begin{figure}[htb]
\centering
\includegraphics[width=0.5\linewidth]{fig/read_Matrix-60}
\end{figure}

这个场景也很有意思,我们首先看到的是摩天大楼的玻璃幕墙反射出来的直升飞机的投影,然后我们同时看到了直升飞机和它的影子,最后我们看到的几乎只有直升飞机。

这里,用镜子投影和真实事物两个事物的对比,来说明电影里阐述的对“真像”的探寻过程。我们先看到的是镜子里的投影——“假的”,然后我们同时看到了真实的事物和投影——“真假共存”,最后我们排除虚假影像的干扰,认识到了真实的事物。

Smith又开始发表演说了,呵呵,我简单归纳一下,免得大家看得睡着:

1、曾经有过第一代的Matrix,没人受苦(完美世界),但是失败了,里面的人都死了。

2、人类自己定义的真实,肯定包含了“受苦”这个元素,靠程序不可能完全模拟出来。

3、新的、就是现在的Matrix是根据人类文明的“顶峰”(20世纪末)来设计的。

4、他提到了“进化”的过程,拿恐龙和人来比,实际上就是说人和恐龙一样要(或者已经)完蛋了。

\begin{figure}[htb]
\centering
\includegraphics[width=0.5\linewidth]{fig/read_Matrix-61}
\end{figure}

他在看窗外时,我们看到下面的场景和Mouse的模拟程序是同样的地方。

\begin{figure}[htb]
\centering
\includegraphics[width=0.5\linewidth]{fig/read_Matrix-62}
\end{figure}

电影里干探出现的地方总是有5,比如一开始Trinity被追踪到的电话号码是555-0690,这里的椅子也是5$\times$5的洞。你可能会说我太过敏了,但是我觉得电影给了这张椅子3秒钟的特写,自然有它的含义。

\begin{myquote}
(Nebuchadnezzar)

Neo: What are they doing to him?

Tank: Breaking into his mind. It's like hacking into a computer, all it takes is time.

Neo: How much time?

Tank: Depends on the mind. Eventually it will crack and his alpha patterns will change from this to this. When it does, Morpheus will tell them anything they want to know.

Neo: Well, what do they want?

Tank: The leader of every ship is given codes to Zion's mainframe computer. If an agent got the codes and got into Zion's mainframe, they could destroy us. We can't let that happen.

Neo: Trinity.

Tank: Zion's more important than me or you or even Morpheus.

Neo: Well, there has to be something that we can do.

Tank: There is. We pull the plug.

Trinity: You're going to kill him? Kill Morpheus?

Tank: We don't have any other choice.
\end{myquote}

这里说的是为什么干探要Zion的主机电脑的密码,为了摧毁Zion。从第二、三部来看,机器根本不需要这个密码,这不过是the One之路中的一个安排罢了。

\begin{myquote}
(Office)

Agent Smith: Never send a human to do a machine's job.

Agent Brown: If indeed the insider has failed, they'll sever the connection as soon as possible, unless ...

Agent Jones: They're dead, in either case ...

Agent Smith: We have no choice but to continue as planned. Deploy the sentinels immediately.

(Nebuchadnezzar)

Tank: Morpheus, you're more than a leader to us. You're our father. We'll miss you always.

Neo: Stop. I don't believe this is happening.

Tank: Neo, this has to be done.

Neo: Does it? I don't know, I ... this can't be just coincidence. It can't be.

Tank: What are you talking about?

Neo: The Oracle. She told me this would happen. She told me that I would have to make a choice.

Trinity: What choice? ... what are you doing?

Neo: I'm going in.

Trinity: No, you're not.

Neo: I have to.

Trinity: Neo, Morpheus sacrificed himself so that he could get you out. There's no way that you're going back in.

Neo: Morpheus did what he did because he believed I am something I'm not.

Trinity: What?

Neo: I'm not the One, Trinity. The Oracle hit me with that too.

Trinity: No, you have to be.

Neo: Sorry, I'm not. I'm just another guy.

Trinity: No, Neo. That's not true. It can't be true.

Neo: Why?

Tank: Neo, this is loco. They've got Morpheus in a military controlled building. Even if you somehow got inside, those are agents holding him. Three of them. I want Morpheus back too, but what you're talking about is suicide.

Neo: I know that's what it looks like, but it's not. I can't explain to you why it's not. Morpheus believed something and he was ready to give his life for what he believed. I understand that now. But that's why I have to go.

Tank: Why?

Neo: Because I believe in something.

Trinity: What?

Neo: I believe I can bring him back ... what are you doing?

Trinity: I'm going with you.

Neo: No, you're not.

Trinity: No? Let me tell you what I believe. I believe Morpheus means more to me than he does to you. I believe if you were really serious about saving him, you are going to need my help. And since I am the ranking officer on this ship, if you don't like, I believe you can go to hell. Because you aren't going anywhere else. Tank, load us up.
\end{myquote}

在他们即将拔Morpheus插头,杀死Morpheus,杀死梦想之时,Neo想起了Oracle的话,他开始相信Oracle的话了。Oracle告诉Neo他可以救回Morpheus,所以他决定接入Matrix。

\begin{myquote}
(Office)

Agent Smith: I'd like to share a revelation during my time here. It came to me when I tried to classify your species. I realized that you're not actually mammals. Every mammal on this planet instinctively develops a natural equilibrium with the surrounding environment, but you humans do not. You move to an area and you multiply and multiply, until every natural resource is consumed. The only way you can survive is to spread to another area. There is another organism on this planet that follows the same pattern. Do you know what it is? A virus. Human beings are a disease, a cancer of this planet. You are a plague, and we are the cure.
\end{myquote}

另外一段非常有趣的话,Smith说人就是病毒(飞船都长得像艾滋病病毒),人类就是不停地复制,浪费地球的资源。机器能拯救这种情况。听起来残酷,却是非常正确。

\begin{myquote}
(Cellular)

Tank: Okay. What do you need, besides a miracle?

Neo: Guns. Lots of guns.

(Construct)

Trinity: Neo, no one has ever done anything like this.

Neo: That's why it's going to work.

(Office)

Agent Smith: Why isn't this serum working?

Agent Brown: Perhaps we're asking the wrong questions.

Agent Smith: Leave me with him. Now.
\end{myquote}

很多人怀疑第二,三集是否和第一部有连续性,或者说是否是凭空写出来的续集。从这里开始我们就应该能发现Smith的特殊性,他不仅仅是一个被干掉的干探那么简单,他的意识和行为显然和其它干探不同。这就是在给后来的剧情做安排。

\begin{myquote}
(Nebuchadnezzar)

Tank: Hold on, Morpheus. They're coming for you. They're coming.

(Office)

Agent Smith: Can you hear me, Morpheus? I'm going to be honest with you. I hate this place, this zoo, this prison, this reality, whatever you want to call it. I can't stand it any longer. It's the Hell, if there is such a thing. I feel saturated by it. I can taste your stink. And every time I do feel I have somehow been infected by it. It's repulsive, isn' t it? I must get out of here. I must get free and in this mind is the key, my key. Once Zion is destroyed, there is no need for me to be here, don't you understand? I need the codes. I have to get inside Zion, and you have to tell me how. You're going to tell me or you're going to die.
\end{myquote}

Smith更加奇怪了,他摘下了耳机和墨镜。电影里摘下墨镜,代表的是一种坦白、真诚的态度。归纳一下他的话:

1、他很讨厌Matrix,他想“自由”,同时他似乎也对于自己“人化”的过程很反感,就像 Animatrix里那个被“同化”的机器人经历的一样。

2、Zion被摧毁,他就要被删除。为什么?他是为了第六代Matrix开发的程序,如果Zion被毁灭(像前几次一样),他的目的完成,自然要被删除。而在Zion的电脑里,他可以像病毒一样继续“活着”。

所以Smith这么想要Zion主机电脑的密码,是“为了自己”。

\myparsep

\begin{myquote}
(Lobby)

Guard 1: Please remove any metallic items you're carrying, keys, loose change ... holy shit.

Guard 2: Backup. Send backup.

Soldier: Freeze!
\end{myquote}

这段在大厅里的枪战是三集里我最喜欢的一段枪战,里面的慢动作,高速摄像机拍出的爆炸场面,超酷的动作,加上动感的音乐,太完美了~!!!

\begin{figure}[htb]
\centering
\includegraphics[width=0.5\linewidth]{fig/read_Matrix-63}
\end{figure}

电影用慢动作捕捉子弹横飞的镜头,不仅仅是耍酷,同时也是在表现那种“雨”的感觉。雨在Matrix里代表一种“变化”、“进化”、“革命”。

\begin{figure}[htb]
\centering
\includegraphics[width=0.5\linewidth]{fig/read_Matrix-64}
\end{figure}

他们打完了,电影给了一个战场的特写。我们看到了两排打得稀烂的柱子,仔细地看,我觉得它们代表的是6代Matrix。左边第一个柱子破坏得最厉害,2、3、4代(左2--4)都是“中度伤害”,5代(右1)基本没事,6代(右2)再次受到很大的破坏。

有朋友会问,Architect不是说,第二代的Matrix也失败了吗?我想这就是导演在写第二、三部剧本时修改的地方了,法国人这个角色的加入和这里也很有关系。

导演实际上是在说明Matrix为什么要采用现在这种1999年代的背景。第一代的共产主义社会失败了,为什么我们不能采用暴政镇压式的管理方式来管理Matrix,谁反抗就干掉谁呢?

第二部告诉我们,第二代的Matrix失败,机器才让人类自己给自己建Matrix。我们活在自己能接受的一个监狱里,习惯了就接受了。

我个人猜测法国人和这个版本的Matrix有关,因为他的名字Merovingian代表的那个法国国王,就是以暴政著称。而且他以“耶稣的后人”自称(实际上就是说代表机器“上帝”),他与Oracle的矛盾和差异,都能说明这个问题。

导演在写第一部剧本时的意思,应该是说曾经有过第一代的Matrix过于完美而失败,然后它们采用了1999年的背景和the One来升级系统,系统越来越完美,到第五代己经几乎没有瑕疵。但是Oracle,有远见的Oracle设计了Neo这个特别的the One,存心要给现在的Matrix带来毁灭,所以第六代Matrix同样被重创。

\begin{myquote}
(Office)

Agent Brown: What were you doing?

Agent Jones: He doesn't know.

Agent Smith: Know what?

Agent Brown: I think they're trying to save you.

(Elevator)

Neo: There is no spoon.
\end{myquote}

Neo说:“There is no spoon.”(勺子不存在)意思是对于他来说,Matrix没有规则的限制。Cypher对他说的规则“看到干探就跑路”,他现在要去打破它。

\begin{figure}[htb]
\centering
\includegraphics[width=0.5\linewidth]{fig/read_Matrix-65}
\end{figure}

注意这个爆炸镜头,我看了第一部的D9里的特效介绍部分,这个门是后来故意加上去的。导演在用这个门来说明,“地狱之门”己经被打开了,烈火吞没了大厅(前面说的柱子就在这里),代表对Matrix的“革命”开始了,全新的Matrix将要诞生。

\begin{myquote}
(Office)

Agent Smith: Find them and destroy them.

(Rooftop)

Pilot: I repeat, we are under attack.

Neo: Trinity, help.

Agent Brown: Only human.

Trinity: Dodge this ... how did you do that?

Neo: Do what?

Trinity: You moved like they do. I've never seen anyone move that fast.

Neo: Wasn't fast enough. Can you fly that thing?

Trinity: Not yet.
\end{myquote}

屋顶Neo躲子弹的镜头相当经典,不重复了。说两个不常被提起的地方。

1、看下面的图,红色的使用,导演故意把后面的水龙头漆成了红色,而且只是头部。

\begin{figure}[htb]
\centering
\includegraphics[width=0.5\linewidth]{fig/read_Matrix-66}
\end{figure}

2、关于“子弹时间”,我想在这里详细说一下:子弹时间,是导演兄弟想出来的概念,他们手下的人完全是用摄像机实现的,没有用到电脑特效。它代表的意思不仅仅是改变时间的速度,而是在说我们身边的规则都是可以改变的。

电影里说到改变规则,用了两个词,“bend”与“break”。break好理解,就是打破的意思;而bend呢,是弯曲的意思,具体来说,子弹时间属于“被弯曲的规则”。时间可以变慢、变快,但是我们却不能“打破它”(脱离时间)。感兴趣的朋友可以去看看相对论之类的科普文章,有详细讨论bend the time的部分。

\begin{myquote}
(Cellular)

Tank: Operator.

Trinity: Tank, I need a pilot program for a B-212 helicopter. Hurry ... let's go.

(Office)

Agent Smith: No.

(Helicopter)

Neo: Morpheus, get up. Get up. Get up ... he's not going to make it ... gotcha.
\end{myquote}

\begin{figure}[htb]
\centering
\includegraphics[width=0.5\linewidth]{fig/read_Matrix-67}
\end{figure}

Neo在直升飞机上扫射时,子弹壳像下雨一样掉了下来。

\begin{figure}[htb]
\centering
\includegraphics[width=0.5\linewidth]{fig/read_Matrix-68}
\end{figure}

这个镜头的特写,给我们的感觉,Morpheus就像在一个洞中。他挣脱手铐,冲出来的镜头,就像柏拉图的“洞穴理论”里的人。

而且Neo也叫了三次get up,对应前面Trinity对自己叫的三次get up。

\begin{myquote}
(Rooftop)

\begin{figure}[htb]
\centering
\includegraphics[width=0.5\linewidth]{fig/read_Matrix-70}
\end{figure}

Neo: Trinity.

(Nebuchadnezzar)

Tank: I knew it. He's the One.

(Rooftop)

Morpheus: Do you believe it now, Trinity?

Neo: Morpheus, the Oracle, she told me I'm ...

Morpheus: She told you exactly what you needed to hear, that's all. Sooner or later you're going to realize, just as I did, there's a difference between knowing the path and walking the path.
\end{myquote}

直升飞机撞大楼这段,镜头是有含义的。

首先,我们看到的是,大楼外墙玻璃反射出来的Matrix世界,直升飞机撞上去,爆炸,代表他们这些黑客在破坏Matrix。

接着,有意思的是,当Trinity撞回这边的大楼时,玻璃破了,像是一个蜘蛛网。代表的意思是,他们以为打破Matrix就自由了,其实是陷入另外一种“控制”。

Morpheus再次重复了他的话:Oracle跟你说的话,只是说给你听的。他说自己只是一个“知道路”的人,而Neo才是要“走那条路”的人。

\myparsep

\begin{myquote}
(Cellular)

Tank: Operator.

Morpheus: Tank.

Tank: God damn. It's good to hear your voice, sir.

Morpheus: Need an exit.

Tank: Got one ready. Subway station, State and Balboa.

(Rooftop)

Agent Smith: Damn it.

Agent Brown: The trace was completed.

Agent Jones: We have their position.

Agent Brown: The sentinels are standing by.

Agent Jones: Order the strike.

Agent Smith: They're not out yet.

(Subway station)

Neo: You first, Morpheus.

Trinity: Neo, I want to tell you something, but I'm afraid of what it could mean if I do. Everything the Oracle told me has come true. Everything but this.

(Nebuchadnezzar)

Trinity: Neo.

Tank: What just happened?

Trinity: An agent. You have to send me back.

Tank: I can't.

(Subway station)

Agent Smith: Mr.~Anderson.

(Nebuchadnezzar)

Trinity: Run, Neo, run! What is he doing?

Morpheus: He's beginning to believe.
\end{myquote}

地铁站,Neo和Smith的决战开始了。

\begin{figure}[htb]
\centering
\includegraphics[width=0.5\linewidth]{fig/read_Matrix-71}
\end{figure}

Neo看了看出口,他选择了回头和干探打,这就是Morpheus之前说的:“总有一天你要面对干探,并击败他们。”“因为你是the One。”

\begin{figure}[htb]
\centering
\includegraphics[width=0.5\linewidth]{fig/read_Matrix-72}
\end{figure}

这个场景居然有京剧的鼓点,而且看后来的花絮,导演还故意安排了两个人躲在后面扔报纸,大概这是哪个港产武打片的镜头。

\begin{figure}[htb]
\centering
\includegraphics[width=0.5\linewidth]{fig/read_Matrix-73}
\end{figure}

他们两个的打斗几乎都是围绕着这个3号站台展开的,“3”又来了。

\begin{myquote}
(Subway station)

Agent Smith: You're empty.

Neo: So are you.

Agent Smith: I'm going to enjoy watching you die, Mr.~Anderson.
\end{myquote}

他们两个打斗的招式相似,说话都像镜子投影。Smith说“You're empty”(你没子弹了),意思也是说Neo准备好“载入”新的代码了。

Smith说:“我要开心地看着你死了,Anderson先生。”没错,Anderson要死了,但the One活过来了。

\begin{myquote}
(Nebuchadnezzar)

Trinity: Jesus, he's killing him.

(Subway station)

Agent Smith: Do you hear that, Mr.~Anderson? that is the sound of inevitability. that is the sound of your death. goodbye, Mr.~Anderson.

Neo: My name is Neo.
\end{myquote}

\begin{figure}[htb]
\centering
\includegraphics[width=0.5\linewidth]{fig/read_Matrix-74}
\end{figure}

Neo的这个手势,我不大记得了,是不是黄飞鸿里面的?反正在第三部里也来了一次,电影一直想告诉我们,故事是一个循环。

\begin{figure}[htb]
\centering
\includegraphics[width=0.5\linewidth]{fig/read_Matrix-75}
\end{figure}

Smith说sound of inevitability(不可避免之声),指的就是火车刹车的发出的那种吱吱声,就像我们拿刀子划玻璃的声音。这种声音贯穿全片,包括一开始Neo被老板训话时,外面工人擦玻璃的声音。

\begin{figure}[htb]
\centering
\includegraphics[width=0.5\linewidth]{fig/read_Matrix-76}
\end{figure}

来的火车在第三部也出现了,loop(循环),第一部里它就来了两次。Trinity刚刚要说话,它就来了,Neo和Smith打得差不多了,它又来了。

\myparsep

\begin{myquote}
(Nebuchadnezzar)

Trinity: What happened?

Tank: I don't know. I lost him. Oh, shit.

Trinity: Sentinels. how long?

Morpheus: Five, maybe six minutes. Tank, charge the EMP.

Trinity: You can't use that until he's out.

Morpheus: I know, Trinity. Don't worry. He's going to make it.
\end{myquote}

\begin{figure}[htb]
\centering
\includegraphics[width=0.5\linewidth]{fig/read_Matrix-77}
\end{figure}

5只章鱼,记得我前面说过的吗,章鱼就是现实世界里的干探,它们都是用5来代表的。

\begin{myquote}
(Street)

Man: Shit, that's my phone! That's my best phone!

(Cellular)

Tank: That's unknown.

Neo: Mr.~wizard, get me the hell out of here.

Tank: Got a patch on an old exit, Wabash and Lake.

Neo: Oh, shit ... help. Need a little help.

Tank: Door ... door on your left. No, you're other left ... back door.
\end{myquote}

\begin{figure}[htb]
\centering
\includegraphics[width=0.5\linewidth]{fig/read_Matrix-78}
\end{figure}

Neo在跑路时,经过一个菜市场。大家注意到了吗,这个菜市场五颜六色的,不只是水果蔬菜, 连路人的衣服也是花花绿绿的,和我们之前看到的Matrix世界完全不同。

后面讲那个hotel招牌时来解释这个事。

\begin{figure}[htb]
\centering
\includegraphics[width=0.5\linewidth]{fig/read_Matrix-79}
\end{figure}

这是Neo被追杀,经过一个厨房时,那个干探“上身”大妈,扔过来的菜刀。大家注意到了吗,不管是菜刀还是子弹,统统都是打在门框上。说干探要杀他,还不如说他们在给Neo指路,或者说把Neo往一个地方赶。

\begin{figure}[htb]
\centering
\includegraphics[width=0.5\linewidth]{fig/read_Matrix-80}
\end{figure}

Neo跳进一个垃圾堆,这也符合耶稣诞生在马槽的故事内容。就像《肖申克的救赎》里的安迪,逃出监狱是从下水道走的。重生的过程往往很艰苦。

\begin{myquote}
(Nebuchadnezzar)

Trinity: Oh, no.

Morpheus: Here they come ... he's going to make it.

(Cellular)

Tank: Fire escape at the end of the alley. Room 303.
\end{myquote}

Neo最后被引向了电影一开始Trinity呆过的303房,同样的旅馆。这里有一个很重要的部分要和大家分享——旅馆的招牌。

\begin{figure}[htb]
\centering
\includegraphics[width=0.5\linewidth]{fig/read_Matrix-81}
\end{figure}

\begin{figure}[htb]
\centering
\includegraphics[width=0.45\linewidth]{fig/read_Matrix-82}
\includegraphics[width=0.45\linewidth]{fig/read_Matrix-83}

\vspace{3pt}

\includegraphics[width=0.45\linewidth]{fig/read_Matrix-84}
\includegraphics[width=0.45\linewidth]{fig/read_Matrix-85}
\end{figure}

1、电影一开始,旅馆的招牌因为在夜幕下,显得是白底黑字,到了这时,天亮了,我们看到了,字是绿色的。绿色是代码的颜色,说明这是机器安排好的场景。

2、下面的两个小招牌:“hourly rates”和“free TV”。

\begin{figure}[!h]
\centering
\includegraphics[width=0.5\linewidth]{fig/read_Matrix-86}
\end{figure}

Hourly rates在开始和结尾都有出现,字面意思是“按小时计费”。如果我们把rates颠倒一下次序(就像第三部的limbo),我们得到了stare(凝视)。用hourly的另外一个意思——每小时,就是“时刻监视”。所以联系电影一开始,这个招牌的意思是Trinity在时时刻刻监视着Neo。

Free TV只出现在最后,字面意思很简单,“免费电视”。TV我之前解释过,tele-vision, 远见、领悟的意思。free代表的是自由,而且这里的free是五颜六色的字体,意思当然是“自由”,代表的是你有了TV(远见)之后,你看到的Matrix就是多彩的。这也是Neo之前跑过的菜市场的含义,代表他在“开眼”了。

下面一段给计算机从业人员。招牌左边的灯泡我按照亮 = 1,不亮 = 0,排列了一下:1111,1111,1111,0110,1011,0111。转换成十六进制,fff6b7 = 3f6b7 = 3rd floor, 6th Matrix become 7th Matrix。想出这个以后,我真的觉得导演兄弟太强了!

\begin{figure}[htb]
\centering
\includegraphics[width=0.5\linewidth]{fig/read_Matrix-87}
\end{figure}

这个旅馆的墙上写着绿色的“4”。前面说了,4代表死亡,绿色代表机器安排。

\begin{myquote}
(Nebuchadnezzar)

Tank: They're inside.

Trinity: Hurry, Neo!

Morpheus: Can't be.

(Hotel)

Agent Smith: Check him.

Agent Brown: He's gone.

Agent Smith: Goodbye, Mr.~Anderson.

(Nebuchadnezzar)

Trinity: Neo, I'm not afraid anymore. The Oracle told me that I would fall in love, and that man, the man who I loved would be the One. And so you see, you can't be dead. You can't be, because I love you. You hear me? I love you ... now get up.
\end{myquote}

Mr.~Anderson死了,Neo得到了他的来生。

\begin{figure}[htb]
\centering
\includegraphics[width=0.5\linewidth]{fig/read_Matrix-88}
\end{figure}

Morpheus要引爆EMP了,我们看到EMP是三个按钮的,而且上面还加了一个数字3。

\begin{figure}[htb]
\centering
\includegraphics[width=0.5\linewidth]{fig/read_Matrix-89}
\end{figure}

Trinity这里说了Oracle的预言:“Oracle告诉我,我会坠入爱河,我爱的那个人就是the One。”她让Neo醒来的“咒语”就是“I love you”,而且她也说了“get up”,前面Trinity对自己说了3次,Neo对Morpheus说了3次,但是这里对Neo只说了一次,因为Neo是“1”?

而且他们接吻的场面,跟法国人老婆和Neo接吻时一样,背景就像在下雨。

\begin{myquote}
(Hotel)

Neo: No.

(Nebuchadnezzar)

Tank: How?

Morpheus: He is the One.

Trinity: Neo.

(Phone)

Neo: I know you re out there. I can feel you now. I know that you're afraid. You're afraid of us. You're afraid of change. I don't know the future. I didn't come here to tell you how it's going to end. I came here to tell you how it's going to begin. I'm going to hang up this phone, and then I'm going to show these people what you don't want them to see. I'm going to show them a world without you, a world without rules and controls, without borders or boundaries, a world where anything is possible. Where we go from there is a choice I leave to you.
\end{myquote}

Neo说:“我来这里不是要告诉你这一切将如何结束,而是要告诉你如何开始。”“去向何方是我留给你的选择。”这些实际上是在给第二、三部留伏笔。

\newpage

\begin{figure}[htb]
\centering
\includegraphics[width=0.45\linewidth]{fig/read_Matrix-90}
\includegraphics[width=0.45\linewidth]{fig/read_Matrix-91}
\end{figure}

说个有趣的东西,最后的画面结束在“SYSTEM FAILURE”中间,准确地说,在M和F之间,M = Matrix,F = Freedom,明白了吧。

呵呵,终于结束了,很辛苦的三个星期,希望看完它的你能有所收获。2和3的同样“解释”我会找时间来完成的,但是看来近期没戏。喜欢Matrix就像一条坎坷的路,走起来还真有些辛苦。

\myparsep

哈,你真强啊,能看完这个长篇废话。还是那句话,这东西是留给“喜欢Matrix的朋友”的。我这里写的东西,大概能有1/3是对的就很好了,而我说的东西大概只是电影的1/10不到,所以就是说这里的东西只是3\%不到的电影解释。写这个东西的用意更多的是在于抛砖引玉,引发更多对于Matrix的思考。

如果你有什么问题想提出来讨论,或者有批评建议,可以用下面几个方式找到我:

Email:matrix.oracle@gmail.com

QQ:22399(基本在线但是不可见,有事留言)

\rightline{neverwin 2004-01}

\myparsep

没错,我就是neverwin大神所说的“喜欢Matrix的朋友”之一。我上小学的时候就在百度贴吧看到了许多大神对Matrix的分析,这些文章很大程度上改变了我的世界观。但是由于时间久远,这些文章散落在网络的各个角落,而且排版让人不忍直视。现在我打算把这些文章重新整理一遍,我也不知道要花多久,反正是兴趣使然吧。

Matrix的第一部,离现在已经快20年了。在这20年里,可以说没有一部电影能够做到设计这么多“有深度”的细节。Matrix的这些细节就像猜迷一样,如果你猜不出来,看到的就只是虚拟世界、人工智能、超能力躲子弹这样一部炫酷的科幻动作片;但是你一旦猜出谜底,就会{\bf 卧槽}。既然你把这篇文章看完了,应该卧槽了好几次吧。很遗憾,Wachowski兄弟(现在是姐妹了= =)之后的《云图》《木星上行》等作品再也没有达到这样的高度。传说华纳要拍Matrix新三部曲,后来也没有动静。

这个项目的Github主页:https://github.com/martinwu42/read-matrix

我的Email:woctordho@outlook.com

\rightline{woctordho 2017-01-26}

\end{document}
